\documentclass{beamer}

\usepackage[UKenglish]{babel}
\usepackage[utf8]{inputenc}
\usepackage{xcolor}
\usepackage{siunitx}
\usepackage[T1]{fontenc}

\usepackage{mathtools}
\usepackage{amsmath}
\usepackage{amssymb}
\usepackage{amsthm}
\usepackage{physics}
\usepackage{nicefrac}
\usepackage[backend=biber]{biblatex}

\addbibresource{../tex/bibliography.bib}

\newcommand{\numberset}{\mathbb}
\newcommand{\N}{\numberset{N}}
\newcommand{\R}{\numberset{R}}
\newcommand{\Q}{\numberset{Q}}
\newcommand{\Z}{\numberset{Z}}
\newcommand{\deq}{\overset{d}{=}}
\newcommand{\etal}{\textit{et al.}}

\graphicspath{{../sunSpots/img/}}

\defbeamertemplate*{title page}{customized}[1][]
{
	\begin{beamercolorbox}[center]{institute}
	\includegraphics[height=.2\textheight]{../tex/10000000000004E1000001E114DE41E38E5F0A34.jpg}
	\par\tiny{DEPARTMENT OF PHYSICS}
	\par\tiny{BACHELOR DEGREE IN PHYSICS}
	\end{beamercolorbox}
	\bigskip
	\begin{beamercolorbox}[sep=0.3cm,center]{title}
		\usebeamerfont{title}An investigation of\\HURST EXPONENT
	\end{beamercolorbox}
	\bigskip
	\begin{beamercolorbox}[center]{author}
	\begin{columns}
		\begin{column}{0.47\textwidth}
			\centering
			Candidate:\\\textbf{Alessandro Foradori}
		\end{column}
	
		\begin{column}{0.47\textwidth}
			\centering
			Supervisor:\\\textbf{Leonardo Ricci}
		\end{column}
	\end{columns}
	\end{beamercolorbox}
	\vfill
	\begin{beamercolorbox}[left]{date}
		\centering
		\small17th November 2020
	\end{beamercolorbox}
}

\setbeamertemplate{navigation symbols}{}
\setbeamertemplate{footline}
{
	\leavevmode%
	\hbox{%
		\begin{beamercolorbox}[wd=.9\paperwidth,ht=2.25ex,dp=1ex,center]{title in head/foot}%
			\usebeamerfont{title in head/foot}
		\end{beamercolorbox}%
		\begin{beamercolorbox}[wd=.1\paperwidth,ht=2.25ex,dp=1ex,center]{date in head/foot}%
			\vfill
			\insertframenumber / \inserttotalframenumber \hspace*{1ex}
	\end{beamercolorbox}}%
	\vskip0pt%
}

\usetheme{Pittsburgh}
\usecolortheme{beaver}
\begin{document}	

{	
\setbeamertemplate{footline}{} 
\begin{frame}
	\titlepage
\end{frame}
}
\addtocounter{framenumber}{-1}

{
	\setbeamertemplate{footline}
	{
		\leavevmode%
		\hbox{%
			\begin{beamercolorbox}[wd=.9\paperwidth,ht=2.25ex,dp=1ex,left,leftskip=2ex]{title in head/foot}
				\begin{itemize}
					\item \fullcite{PhysRevE.49.1685}
					\item \fullcite{Foradori}
				\end{itemize}
			\end{beamercolorbox}%
			\begin{beamercolorbox}[wd=.1\paperwidth,ht=2.25ex,dp=1ex,center]{date in head/foot}%
				\vfill
				\insertframenumber / \inserttotalframenumber \hspace*{1ex}
		\end{beamercolorbox}}%
		\vskip0pt%
	}

\begin{frame}{Introduction}
	\begin{block}{Hurst exponent}
		The Hurst exponent is a dimensionless estimator used to evaluate self-similarity and long-range dependence properties of time series. 
		\begin{equation*}
			\mathbb{E} \left[\frac{R(N)}{S(N)}\right] \propto N^{H} \quad \text{for } N \to \infty
		\end{equation*}
	\end{block}

	\begin{block}{Detrended fluctuation analysis}
	\end{block}

\end{frame}
}

{
\setbeamertemplate{footline}
	{
		\leavevmode%
		\hbox{%
			\begin{beamercolorbox}[wd=.9\paperwidth,ht=2.25ex,dp=1ex,left,leftskip=2ex]{title in head/foot}
				\begin{itemize}
					\item \fullcite{pipiras_taqqu_2017}
					\item \fullcite{Beran:2304008}
				\end{itemize}
			\end{beamercolorbox}%
			\begin{beamercolorbox}[wd=.1\paperwidth,ht=2.25ex,dp=1ex,center]{date in head/foot}%
				\vfill
				\insertframenumber / \inserttotalframenumber \hspace*{1ex}
		\end{beamercolorbox}}%
		\vskip0pt%
	}
\begin{frame}{Long Memory}
		\begin{block}{Time series}
		A discrete weakly stationary stochastic process $\{X_n\}_{n \in \Z}$ is called \textbf{time series}.
	\end{block}
\end{frame}
}

\begin{frame}{Long memory}
	\begin{block}{Long-range dependence}
		\begin{itemize}
			\item $\gamma_X(k) \propto	k^{2d -1}$
			\item $\sum_{k} \abs{\gamma_X(k)} = \infty$
			\item $\textnormal{Var}(X_1 + \dots + X_N) \propto N^{2d + 1}$
		\end{itemize}
		with $ d \in (0, 0.5) $
	\end{block}
	\begin{block}{Short-range dependence}
		$\sum_{k} \abs{\gamma_X(k)} < \infty$
	\end{block}
	\begin{block}{Antipersistance}
		$\sum_{k} \abs{\gamma_X(k)} = 0$ and $d < 0$
	\end{block}
\end{frame}


\begin{frame}{Long memory}
	\begin{block}{Self-similarity}
		Intuitively, self-similarity means that a stochastic process scaled in time (that is plotted with a different time scale) looks statistically the same as the original process when properly rescaled in space. 
		\begin{equation*}
			\{ X(ct) \}_{t \in \R} \deq \{ c^H X(t) \}_{t \in \R}  \quad \text{with $H>0$ and $c>0$}		
		\end{equation*} 
	\end{block}
\end{frame}

\begin{frame}{Long memory}
	\begin{equation*}
		H = \frac{1}{2} + d
	\end{equation*}
	\begin{itemize}
		\item $H \in (0.5, 1]$ long-range dependence
		\item $H = 0.5$ no time lag
		\item $H \in (0, 0.5)$ anti-persistent short-range dependence 
	\end{itemize}
\end{frame}

{
	\setbeamertemplate{footline}
	{
		\leavevmode%
		\hbox{%
			\begin{beamercolorbox}[wd=.9\paperwidth,ht=2.25ex,dp=1ex,left,leftskip=2ex]{title in head/foot}
				\begin{itemize}
					\item \fullcite{pipiras_taqqu_2017}
					\item \fullcite{Beran:2304008}
				\end{itemize}
			\end{beamercolorbox}%
			\begin{beamercolorbox}[wd=.1\paperwidth,ht=2.25ex,dp=1ex,center]{date in head/foot}%
				\vfill
				\insertframenumber / \inserttotalframenumber \hspace*{1ex}
		\end{beamercolorbox}}%
		\vskip0pt%
	}

\begin{frame}{DFA}{Purpose}
	\begin{block}{}
		The purpose of DFA is to estimate the variance of partial sums of the series $X = \{ X_n \}_{n \in \Z}$. In this way, $\textnormal{Var}(X_1 + \dots + X_N) \propto N^H$ allows to estimate the Hurst exponent
	\end{block}
\end{frame}
}

\begin{frame}{MF-DFA}{Step1}
	\begin{block}{}
		Define the "profile" $\{Y(t)\}$ of $\{X(t)\}$.
		\begin{equation*}
			Y( t ) = \int_{0}^{t} dt' (X( t' ) - \bar{X} ) 
		\end{equation*}
		where $\bar{X}$ is the mean of $\{X(t)\}$ computed on the whole time series.
	\end{block}
\end{frame}

\begin{frame}{MF-DFA}{Step2}
	\begin{block}{}
			Split the full time $T$ in time windows $\tau$.
	\end{block}
\end{frame}

\begin{frame}{MF-DFA}{Step3}
	\begin{block}{}
		Compute the linear regression $ g^{( k )}(t, \tau, t') $ of order $k$ of $\{Y(t')\}$ in range $t' \in \left[t, t + \tau \right]$.
		\\
		Determinate the variance $ f_2^{( k )}( t, \tau ) $ between the profile $\{Y(t)\}$ and the regression $ g^{( k )}(t, \tau, t') $.
		\begin{equation*}
			f_{2}^{( k )}( t, \tau ) \equiv \frac{1}{\tau} \int_{t}^{t+\tau} dt' \left( Y( t' ) - g^{( k )}(t, \tau, t')  \right)^2
		\end{equation*}
	\end{block}
\end{frame}

\begin{frame}{MF-DFA}{Step4}
	\begin{block}{}
		$F_q^{( k )}(\tau )$ is the momentum of order $q$ over all time windows of length $\tau$.
		\begin{equation*}
			F_q^{( k )} ( \tau ) =  \left(\int_0^{T-\tau} dt \left( f_2^{( k )}( t, \tau ) \right)^{\nicefrac{q}{2}} p(t, \tau)  \right)^{\nicefrac{1}{q}}
		\end{equation*}
	\end{block}
\end{frame}

\begin{frame}{MF-DFA}{Step5}
	\begin{block}{}
		Determine the scaling behaviour of the fluctuation functions by analysing
		log–log plots of $F_q^{( k )}(\tau )$ versus $\tau$.
		\begin{equation*}
			F_q^{( k )}(\tau ) \propto \tau^{\alpha(q)}
		\end{equation*}
	\end{block}
\end{frame}

\begin{frame}{DFA}{Interpretation}
	\begin{block}{}
		\begin{itemize}
			\item $\alpha(2) \in (0, 1]$ stationary process with $\alpha(2) = H$
			\item $\alpha(2) \in (1, 2]$ non-stationary process with $\alpha(2) = 1 + H$
		\end{itemize}
	\end{block}
\end{frame}

{
	\setbeamertemplate{footline}
	{
		\leavevmode%
		\hbox{%
			\begin{beamercolorbox}[wd=.9\paperwidth,ht=2.25ex,dp=1ex,left,leftskip=2ex]{title in head/foot}
				\begin{itemize}
					\item \fullcite{sidc}
					\item \fullcite{Movahed_2006}
				\end{itemize}
			\end{beamercolorbox}%
			\begin{beamercolorbox}[wd=.1\paperwidth,ht=2.25ex,dp=1ex,center]{date in head/foot}%
				\vfill
				\insertframenumber / \inserttotalframenumber \hspace*{1ex}
		\end{beamercolorbox}}%
		\vskip0pt%
	}

\begin{frame}{Sunspots}
	\begin{block}{}
		On the Sun's photosphere there is a strong magnetic field. However in some localized regions (called \textbf{sunspots}), the field is significant higher and the surface appear as spots darker than the surrounding areas. Their number varies according to the approximately 11-year solar cycle	
	\end{block}
\end{frame}
}

\begin{frame}{Sunspots}{Raw data and profile}
	\begin{columns}
		\begin{column}{.47\textwidth}
			\centering
			\begin{block}{\small{daily sunspot number}}
				\scalebox{.7}{% GNUPLOT: LaTeX picture with Postscript
\begingroup
  \makeatletter
  \providecommand\color[2][]{%
    \GenericError{(gnuplot) \space\space\space\@spaces}{%
      Package color not loaded in conjunction with
      terminal option `colourtext'%
    }{See the gnuplot documentation for explanation.%
    }{Either use 'blacktext' in gnuplot or load the package
      color.sty in LaTeX.}%
    \renewcommand\color[2][]{}%
  }%
  \providecommand\includegraphics[2][]{%
    \GenericError{(gnuplot) \space\space\space\@spaces}{%
      Package graphicx or graphics not loaded%
    }{See the gnuplot documentation for explanation.%
    }{The gnuplot epslatex terminal needs graphicx.sty or graphics.sty.}%
    \renewcommand\includegraphics[2][]{}%
  }%
  \providecommand\rotatebox[2]{#2}%
  \@ifundefined{ifGPcolor}{%
    \newif\ifGPcolor
    \GPcolortrue
  }{}%
  \@ifundefined{ifGPblacktext}{%
    \newif\ifGPblacktext
    \GPblacktexttrue
  }{}%
  % define a \g@addto@macro without @ in the name:
  \let\gplgaddtomacro\g@addto@macro
  % define empty templates for all commands taking text:
  \gdef\gplbacktext{}%
  \gdef\gplfronttext{}%
  \makeatother
  \ifGPblacktext
    % no textcolor at all
    \def\colorrgb#1{}%
    \def\colorgray#1{}%
  \else
    % gray or color?
    \ifGPcolor
      \def\colorrgb#1{\color[rgb]{#1}}%
      \def\colorgray#1{\color[gray]{#1}}%
      \expandafter\def\csname LTw\endcsname{\color{white}}%
      \expandafter\def\csname LTb\endcsname{\color{black}}%
      \expandafter\def\csname LTa\endcsname{\color{black}}%
      \expandafter\def\csname LT0\endcsname{\color[rgb]{1,0,0}}%
      \expandafter\def\csname LT1\endcsname{\color[rgb]{0,1,0}}%
      \expandafter\def\csname LT2\endcsname{\color[rgb]{0,0,1}}%
      \expandafter\def\csname LT3\endcsname{\color[rgb]{1,0,1}}%
      \expandafter\def\csname LT4\endcsname{\color[rgb]{0,1,1}}%
      \expandafter\def\csname LT5\endcsname{\color[rgb]{1,1,0}}%
      \expandafter\def\csname LT6\endcsname{\color[rgb]{0,0,0}}%
      \expandafter\def\csname LT7\endcsname{\color[rgb]{1,0.3,0}}%
      \expandafter\def\csname LT8\endcsname{\color[rgb]{0.5,0.5,0.5}}%
    \else
      % gray
      \def\colorrgb#1{\color{black}}%
      \def\colorgray#1{\color[gray]{#1}}%
      \expandafter\def\csname LTw\endcsname{\color{white}}%
      \expandafter\def\csname LTb\endcsname{\color{black}}%
      \expandafter\def\csname LTa\endcsname{\color{black}}%
      \expandafter\def\csname LT0\endcsname{\color{black}}%
      \expandafter\def\csname LT1\endcsname{\color{black}}%
      \expandafter\def\csname LT2\endcsname{\color{black}}%
      \expandafter\def\csname LT3\endcsname{\color{black}}%
      \expandafter\def\csname LT4\endcsname{\color{black}}%
      \expandafter\def\csname LT5\endcsname{\color{black}}%
      \expandafter\def\csname LT6\endcsname{\color{black}}%
      \expandafter\def\csname LT7\endcsname{\color{black}}%
      \expandafter\def\csname LT8\endcsname{\color{black}}%
    \fi
  \fi
    \setlength{\unitlength}{0.0500bp}%
    \ifx\gptboxheight\undefined%
      \newlength{\gptboxheight}%
      \newlength{\gptboxwidth}%
      \newsavebox{\gptboxtext}%
    \fi%
    \setlength{\fboxrule}{0.5pt}%
    \setlength{\fboxsep}{1pt}%
\begin{picture}(4520.00,2540.00)%
    \gplgaddtomacro\gplbacktext{%
      \csname LTb\endcsname%%
      \put(445,489){\makebox(0,0)[r]{\strut{}$-100$}}%
      \csname LTb\endcsname%%
      \put(445,743){\makebox(0,0)[r]{\strut{}$-50$}}%
      \csname LTb\endcsname%%
      \put(445,997){\makebox(0,0)[r]{\strut{}$0$}}%
      \csname LTb\endcsname%%
      \put(445,1251){\makebox(0,0)[r]{\strut{}$50$}}%
      \csname LTb\endcsname%%
      \put(445,1506){\makebox(0,0)[r]{\strut{}$100$}}%
      \csname LTb\endcsname%%
      \put(445,1760){\makebox(0,0)[r]{\strut{}$150$}}%
      \csname LTb\endcsname%%
      \put(445,2014){\makebox(0,0)[r]{\strut{}$200$}}%
      \csname LTb\endcsname%%
      \put(445,2268){\makebox(0,0)[r]{\strut{}$250$}}%
      \csname LTb\endcsname%%
      \put(526,68){\rotatebox{45}{\makebox(0,0){\strut{}1749}}}%
      \csname LTb\endcsname%%
      \put(933,68){\rotatebox{45}{\makebox(0,0){\strut{}1779}}}%
      \csname LTb\endcsname%%
      \put(1340,68){\rotatebox{45}{\makebox(0,0){\strut{}1809}}}%
      \csname LTb\endcsname%%
      \put(1747,68){\rotatebox{45}{\makebox(0,0){\strut{}1839}}}%
      \csname LTb\endcsname%%
      \put(2154,68){\rotatebox{45}{\makebox(0,0){\strut{}1869}}}%
      \csname LTb\endcsname%%
      \put(2561,68){\rotatebox{45}{\makebox(0,0){\strut{}1899}}}%
      \csname LTb\endcsname%%
      \put(2969,68){\rotatebox{45}{\makebox(0,0){\strut{}1930}}}%
      \csname LTb\endcsname%%
      \put(3376,68){\rotatebox{45}{\makebox(0,0){\strut{}1960}}}%
      \csname LTb\endcsname%%
      \put(3783,68){\rotatebox{45}{\makebox(0,0){\strut{}1990}}}%
      \csname LTb\endcsname%%
      \put(4190,68){\rotatebox{45}{\makebox(0,0){\strut{}2020}}}%
    }%
    \gplgaddtomacro\gplfronttext{%
    }%
    \gplbacktext
    \put(0,0){\includegraphics[width={226.00bp},height={127.00bp}]{../img/fourier/dataPlot}}%
    \gplfronttext
  \end{picture}%
\endgroup
}
				\scalebox{.7}{% GNUPLOT: LaTeX picture with Postscript
\begingroup
  \makeatletter
  \providecommand\color[2][]{%
    \GenericError{(gnuplot) \space\space\space\@spaces}{%
      Package color not loaded in conjunction with
      terminal option `colourtext'%
    }{See the gnuplot documentation for explanation.%
    }{Either use 'blacktext' in gnuplot or load the package
      color.sty in LaTeX.}%
    \renewcommand\color[2][]{}%
  }%
  \providecommand\includegraphics[2][]{%
    \GenericError{(gnuplot) \space\space\space\@spaces}{%
      Package graphicx or graphics not loaded%
    }{See the gnuplot documentation for explanation.%
    }{The gnuplot epslatex terminal needs graphicx.sty or graphics.sty.}%
    \renewcommand\includegraphics[2][]{}%
  }%
  \providecommand\rotatebox[2]{#2}%
  \@ifundefined{ifGPcolor}{%
    \newif\ifGPcolor
    \GPcolortrue
  }{}%
  \@ifundefined{ifGPblacktext}{%
    \newif\ifGPblacktext
    \GPblacktexttrue
  }{}%
  % define a \g@addto@macro without @ in the name:
  \let\gplgaddtomacro\g@addto@macro
  % define empty templates for all commands taking text:
  \gdef\gplbacktext{}%
  \gdef\gplfronttext{}%
  \makeatother
  \ifGPblacktext
    % no textcolor at all
    \def\colorrgb#1{}%
    \def\colorgray#1{}%
  \else
    % gray or color?
    \ifGPcolor
      \def\colorrgb#1{\color[rgb]{#1}}%
      \def\colorgray#1{\color[gray]{#1}}%
      \expandafter\def\csname LTw\endcsname{\color{white}}%
      \expandafter\def\csname LTb\endcsname{\color{black}}%
      \expandafter\def\csname LTa\endcsname{\color{black}}%
      \expandafter\def\csname LT0\endcsname{\color[rgb]{1,0,0}}%
      \expandafter\def\csname LT1\endcsname{\color[rgb]{0,1,0}}%
      \expandafter\def\csname LT2\endcsname{\color[rgb]{0,0,1}}%
      \expandafter\def\csname LT3\endcsname{\color[rgb]{1,0,1}}%
      \expandafter\def\csname LT4\endcsname{\color[rgb]{0,1,1}}%
      \expandafter\def\csname LT5\endcsname{\color[rgb]{1,1,0}}%
      \expandafter\def\csname LT6\endcsname{\color[rgb]{0,0,0}}%
      \expandafter\def\csname LT7\endcsname{\color[rgb]{1,0.3,0}}%
      \expandafter\def\csname LT8\endcsname{\color[rgb]{0.5,0.5,0.5}}%
    \else
      % gray
      \def\colorrgb#1{\color{black}}%
      \def\colorgray#1{\color[gray]{#1}}%
      \expandafter\def\csname LTw\endcsname{\color{white}}%
      \expandafter\def\csname LTb\endcsname{\color{black}}%
      \expandafter\def\csname LTa\endcsname{\color{black}}%
      \expandafter\def\csname LT0\endcsname{\color{black}}%
      \expandafter\def\csname LT1\endcsname{\color{black}}%
      \expandafter\def\csname LT2\endcsname{\color{black}}%
      \expandafter\def\csname LT3\endcsname{\color{black}}%
      \expandafter\def\csname LT4\endcsname{\color{black}}%
      \expandafter\def\csname LT5\endcsname{\color{black}}%
      \expandafter\def\csname LT6\endcsname{\color{black}}%
      \expandafter\def\csname LT7\endcsname{\color{black}}%
      \expandafter\def\csname LT8\endcsname{\color{black}}%
    \fi
  \fi
    \setlength{\unitlength}{0.0500bp}%
    \ifx\gptboxheight\undefined%
      \newlength{\gptboxheight}%
      \newlength{\gptboxwidth}%
      \newsavebox{\gptboxtext}%
    \fi%
    \setlength{\fboxrule}{0.5pt}%
    \setlength{\fboxsep}{1pt}%
\begin{picture}(9060.00,5100.00)%
    \gplgaddtomacro\gplbacktext{%
      \csname LTb\endcsname%%
      \put(520,836){\rotatebox{45}{\makebox(0,0){\strut{}$ -2 \times10^{4}$}}}%
      \csname LTb\endcsname%%
      \put(520,1547){\rotatebox{45}{\makebox(0,0){\strut{}$ -1 \times10^{4}$}}}%
      \csname LTb\endcsname%%
      \put(520,2259){\rotatebox{45}{\makebox(0,0){\strut{}$ -5 \times10^{3}$}}}%
      \csname LTb\endcsname%%
      \put(520,2970){\rotatebox{45}{\makebox(0,0){\strut{}$  0 \times10^{0}$}}}%
      \csname LTb\endcsname%%
      \put(520,3682){\rotatebox{45}{\makebox(0,0){\strut{}$  5 \times10^{3}$}}}%
      \csname LTb\endcsname%%
      \put(520,4394){\rotatebox{45}{\makebox(0,0){\strut{}$  1 \times10^{4}$}}}%
      \csname LTb\endcsname%%
      \put(872,448){\makebox(0,0){\strut{}1749}}%
      \csname LTb\endcsname%%
      \put(1742,448){\makebox(0,0){\strut{}1779}}%
      \csname LTb\endcsname%%
      \put(2611,448){\makebox(0,0){\strut{}1809}}%
      \csname LTb\endcsname%%
      \put(3481,448){\makebox(0,0){\strut{}1839}}%
      \csname LTb\endcsname%%
      \put(4350,448){\makebox(0,0){\strut{}1869}}%
      \csname LTb\endcsname%%
      \put(5220,448){\makebox(0,0){\strut{}1899}}%
      \csname LTb\endcsname%%
      \put(6089,448){\makebox(0,0){\strut{}1930}}%
      \csname LTb\endcsname%%
      \put(6959,448){\makebox(0,0){\strut{}1960}}%
      \csname LTb\endcsname%%
      \put(7829,448){\makebox(0,0){\strut{}1990}}%
      \csname LTb\endcsname%%
      \put(8698,448){\makebox(0,0){\strut{}2020}}%
    }%
    \gplgaddtomacro\gplfronttext{%
      \csname LTb\endcsname%%
      \put(74,2559){\rotatebox{-270}{\makebox(0,0){\strut{}$ Y_n$}}}%
      \csname LTb\endcsname%%
      \put(4787,142){\makebox(0,0){\strut{}t [year]}}%
      \csname LTb\endcsname%%
      \put(2401,1039){\makebox(0,0)[r]{\strut{}815}}%
      \csname LTb\endcsname%%
      \put(2401,835){\makebox(0,0)[r]{\strut{}576}}%
      \csname LTb\endcsname%%
      \put(3490,1039){\makebox(0,0)[r]{\strut{}450}}%
      \csname LTb\endcsname%%
      \put(3490,835){\makebox(0,0)[r]{\strut{}400}}%
      \csname LTb\endcsname%%
      \put(4579,1039){\makebox(0,0)[r]{\strut{}311}}%
      \csname LTb\endcsname%%
      \put(4579,835){\makebox(0,0)[r]{\strut{}280}}%
      \csname LTb\endcsname%%
      \put(5668,1039){\makebox(0,0)[r]{\strut{}220}}%
      \csname LTb\endcsname%%
      \put(5668,835){\makebox(0,0)[r]{\strut{}132}}%
      \csname LTb\endcsname%%
      \put(6757,1039){\makebox(0,0)[r]{\strut{}60}}%
      \csname LTb\endcsname%%
      \put(6757,835){\makebox(0,0)[r]{\strut{}17}}%
      \csname LTb\endcsname%%
      \put(4787,4773){\makebox(0,0){\strut{}profile function for sun spots number}}%
    }%
    \gplbacktext
    \put(0,0){\includegraphics[width={453.00bp},height={255.00bp}]{../img/true/profilePlot}}%
    \gplfronttext
  \end{picture}%
\endgroup
}
			\end{block}
		\end{column}
		\hfill
		\begin{column}{.47\textwidth}
			\centering
			\begin{block}{\small{monthly mean sunspot number}}
				\scalebox{.7}{% GNUPLOT: LaTeX picture with Postscript
\begingroup
  \makeatletter
  \providecommand\color[2][]{%
    \GenericError{(gnuplot) \space\space\space\@spaces}{%
      Package color not loaded in conjunction with
      terminal option `colourtext'%
    }{See the gnuplot documentation for explanation.%
    }{Either use 'blacktext' in gnuplot or load the package
      color.sty in LaTeX.}%
    \renewcommand\color[2][]{}%
  }%
  \providecommand\includegraphics[2][]{%
    \GenericError{(gnuplot) \space\space\space\@spaces}{%
      Package graphicx or graphics not loaded%
    }{See the gnuplot documentation for explanation.%
    }{The gnuplot epslatex terminal needs graphicx.sty or graphics.sty.}%
    \renewcommand\includegraphics[2][]{}%
  }%
  \providecommand\rotatebox[2]{#2}%
  \@ifundefined{ifGPcolor}{%
    \newif\ifGPcolor
    \GPcolortrue
  }{}%
  \@ifundefined{ifGPblacktext}{%
    \newif\ifGPblacktext
    \GPblacktexttrue
  }{}%
  % define a \g@addto@macro without @ in the name:
  \let\gplgaddtomacro\g@addto@macro
  % define empty templates for all commands taking text:
  \gdef\gplbacktext{}%
  \gdef\gplfronttext{}%
  \makeatother
  \ifGPblacktext
    % no textcolor at all
    \def\colorrgb#1{}%
    \def\colorgray#1{}%
  \else
    % gray or color?
    \ifGPcolor
      \def\colorrgb#1{\color[rgb]{#1}}%
      \def\colorgray#1{\color[gray]{#1}}%
      \expandafter\def\csname LTw\endcsname{\color{white}}%
      \expandafter\def\csname LTb\endcsname{\color{black}}%
      \expandafter\def\csname LTa\endcsname{\color{black}}%
      \expandafter\def\csname LT0\endcsname{\color[rgb]{1,0,0}}%
      \expandafter\def\csname LT1\endcsname{\color[rgb]{0,1,0}}%
      \expandafter\def\csname LT2\endcsname{\color[rgb]{0,0,1}}%
      \expandafter\def\csname LT3\endcsname{\color[rgb]{1,0,1}}%
      \expandafter\def\csname LT4\endcsname{\color[rgb]{0,1,1}}%
      \expandafter\def\csname LT5\endcsname{\color[rgb]{1,1,0}}%
      \expandafter\def\csname LT6\endcsname{\color[rgb]{0,0,0}}%
      \expandafter\def\csname LT7\endcsname{\color[rgb]{1,0.3,0}}%
      \expandafter\def\csname LT8\endcsname{\color[rgb]{0.5,0.5,0.5}}%
    \else
      % gray
      \def\colorrgb#1{\color{black}}%
      \def\colorgray#1{\color[gray]{#1}}%
      \expandafter\def\csname LTw\endcsname{\color{white}}%
      \expandafter\def\csname LTb\endcsname{\color{black}}%
      \expandafter\def\csname LTa\endcsname{\color{black}}%
      \expandafter\def\csname LT0\endcsname{\color{black}}%
      \expandafter\def\csname LT1\endcsname{\color{black}}%
      \expandafter\def\csname LT2\endcsname{\color{black}}%
      \expandafter\def\csname LT3\endcsname{\color{black}}%
      \expandafter\def\csname LT4\endcsname{\color{black}}%
      \expandafter\def\csname LT5\endcsname{\color{black}}%
      \expandafter\def\csname LT6\endcsname{\color{black}}%
      \expandafter\def\csname LT7\endcsname{\color{black}}%
      \expandafter\def\csname LT8\endcsname{\color{black}}%
    \fi
  \fi
    \setlength{\unitlength}{0.0500bp}%
    \ifx\gptboxheight\undefined%
      \newlength{\gptboxheight}%
      \newlength{\gptboxwidth}%
      \newsavebox{\gptboxtext}%
    \fi%
    \setlength{\fboxrule}{0.5pt}%
    \setlength{\fboxsep}{1pt}%
\begin{picture}(9060.00,5100.00)%
    \gplgaddtomacro\gplbacktext{%
      \csname LTb\endcsname%%
      \put(708,652){\makebox(0,0)[r]{\strut{}$0$}}%
      \csname LTb\endcsname%%
      \put(708,1087){\makebox(0,0)[r]{\strut{}$50$}}%
      \csname LTb\endcsname%%
      \put(708,1523){\makebox(0,0)[r]{\strut{}$100$}}%
      \csname LTb\endcsname%%
      \put(708,1958){\makebox(0,0)[r]{\strut{}$150$}}%
      \csname LTb\endcsname%%
      \put(708,2394){\makebox(0,0)[r]{\strut{}$200$}}%
      \csname LTb\endcsname%%
      \put(708,2829){\makebox(0,0)[r]{\strut{}$250$}}%
      \csname LTb\endcsname%%
      \put(708,3265){\makebox(0,0)[r]{\strut{}$300$}}%
      \csname LTb\endcsname%%
      \put(708,3700){\makebox(0,0)[r]{\strut{}$350$}}%
      \csname LTb\endcsname%%
      \put(708,4136){\makebox(0,0)[r]{\strut{}$400$}}%
      \csname LTb\endcsname%%
      \put(820,448){\makebox(0,0){\strut{}1749}}%
      \csname LTb\endcsname%%
      \put(1695,448){\makebox(0,0){\strut{}1779}}%
      \csname LTb\endcsname%%
      \put(2571,448){\makebox(0,0){\strut{}1809}}%
      \csname LTb\endcsname%%
      \put(3446,448){\makebox(0,0){\strut{}1839}}%
      \csname LTb\endcsname%%
      \put(4321,448){\makebox(0,0){\strut{}1869}}%
      \csname LTb\endcsname%%
      \put(5197,448){\makebox(0,0){\strut{}1899}}%
      \csname LTb\endcsname%%
      \put(6072,448){\makebox(0,0){\strut{}1930}}%
      \csname LTb\endcsname%%
      \put(6947,448){\makebox(0,0){\strut{}1960}}%
      \csname LTb\endcsname%%
      \put(7823,448){\makebox(0,0){\strut{}1990}}%
      \csname LTb\endcsname%%
      \put(8698,448){\makebox(0,0){\strut{}2020}}%
    }%
    \gplgaddtomacro\gplfronttext{%
      \csname LTb\endcsname%%
      \put(186,2559){\rotatebox{-270}{\makebox(0,0){\strut{}Count}}}%
      \csname LTb\endcsname%%
      \put(4761,142){\makebox(0,0){\strut{}Time [year]}}%
      \csname LTb\endcsname%%
      \put(4761,4773){\makebox(0,0){\strut{}Monthly observed sun spots number}}%
    }%
    \gplbacktext
    \put(0,0){\includegraphics[width={453.00bp},height={255.00bp}]{../img/monthly/dataPlot}}%
    \gplfronttext
  \end{picture}%
\endgroup
}
				\scalebox{.7}{% GNUPLOT: LaTeX picture with Postscript
\begingroup
  \makeatletter
  \providecommand\color[2][]{%
    \GenericError{(gnuplot) \space\space\space\@spaces}{%
      Package color not loaded in conjunction with
      terminal option `colourtext'%
    }{See the gnuplot documentation for explanation.%
    }{Either use 'blacktext' in gnuplot or load the package
      color.sty in LaTeX.}%
    \renewcommand\color[2][]{}%
  }%
  \providecommand\includegraphics[2][]{%
    \GenericError{(gnuplot) \space\space\space\@spaces}{%
      Package graphicx or graphics not loaded%
    }{See the gnuplot documentation for explanation.%
    }{The gnuplot epslatex terminal needs graphicx.sty or graphics.sty.}%
    \renewcommand\includegraphics[2][]{}%
  }%
  \providecommand\rotatebox[2]{#2}%
  \@ifundefined{ifGPcolor}{%
    \newif\ifGPcolor
    \GPcolortrue
  }{}%
  \@ifundefined{ifGPblacktext}{%
    \newif\ifGPblacktext
    \GPblacktexttrue
  }{}%
  % define a \g@addto@macro without @ in the name:
  \let\gplgaddtomacro\g@addto@macro
  % define empty templates for all commands taking text:
  \gdef\gplbacktext{}%
  \gdef\gplfronttext{}%
  \makeatother
  \ifGPblacktext
    % no textcolor at all
    \def\colorrgb#1{}%
    \def\colorgray#1{}%
  \else
    % gray or color?
    \ifGPcolor
      \def\colorrgb#1{\color[rgb]{#1}}%
      \def\colorgray#1{\color[gray]{#1}}%
      \expandafter\def\csname LTw\endcsname{\color{white}}%
      \expandafter\def\csname LTb\endcsname{\color{black}}%
      \expandafter\def\csname LTa\endcsname{\color{black}}%
      \expandafter\def\csname LT0\endcsname{\color[rgb]{1,0,0}}%
      \expandafter\def\csname LT1\endcsname{\color[rgb]{0,1,0}}%
      \expandafter\def\csname LT2\endcsname{\color[rgb]{0,0,1}}%
      \expandafter\def\csname LT3\endcsname{\color[rgb]{1,0,1}}%
      \expandafter\def\csname LT4\endcsname{\color[rgb]{0,1,1}}%
      \expandafter\def\csname LT5\endcsname{\color[rgb]{1,1,0}}%
      \expandafter\def\csname LT6\endcsname{\color[rgb]{0,0,0}}%
      \expandafter\def\csname LT7\endcsname{\color[rgb]{1,0.3,0}}%
      \expandafter\def\csname LT8\endcsname{\color[rgb]{0.5,0.5,0.5}}%
    \else
      % gray
      \def\colorrgb#1{\color{black}}%
      \def\colorgray#1{\color[gray]{#1}}%
      \expandafter\def\csname LTw\endcsname{\color{white}}%
      \expandafter\def\csname LTb\endcsname{\color{black}}%
      \expandafter\def\csname LTa\endcsname{\color{black}}%
      \expandafter\def\csname LT0\endcsname{\color{black}}%
      \expandafter\def\csname LT1\endcsname{\color{black}}%
      \expandafter\def\csname LT2\endcsname{\color{black}}%
      \expandafter\def\csname LT3\endcsname{\color{black}}%
      \expandafter\def\csname LT4\endcsname{\color{black}}%
      \expandafter\def\csname LT5\endcsname{\color{black}}%
      \expandafter\def\csname LT6\endcsname{\color{black}}%
      \expandafter\def\csname LT7\endcsname{\color{black}}%
      \expandafter\def\csname LT8\endcsname{\color{black}}%
    \fi
  \fi
    \setlength{\unitlength}{0.0500bp}%
    \ifx\gptboxheight\undefined%
      \newlength{\gptboxheight}%
      \newlength{\gptboxwidth}%
      \newsavebox{\gptboxtext}%
    \fi%
    \setlength{\fboxrule}{0.5pt}%
    \setlength{\fboxsep}{1pt}%
\begin{picture}(9060.00,5100.00)%
    \gplgaddtomacro\gplbacktext{%
      \csname LTb\endcsname%%
      \put(520,836){\rotatebox{45}{\makebox(0,0){\strut{}$ -2 \times10^{4}$}}}%
      \csname LTb\endcsname%%
      \put(520,1547){\rotatebox{45}{\makebox(0,0){\strut{}$ -1 \times10^{4}$}}}%
      \csname LTb\endcsname%%
      \put(520,2259){\rotatebox{45}{\makebox(0,0){\strut{}$ -5 \times10^{3}$}}}%
      \csname LTb\endcsname%%
      \put(520,2970){\rotatebox{45}{\makebox(0,0){\strut{}$  0 \times10^{0}$}}}%
      \csname LTb\endcsname%%
      \put(520,3682){\rotatebox{45}{\makebox(0,0){\strut{}$  5 \times10^{3}$}}}%
      \csname LTb\endcsname%%
      \put(520,4394){\rotatebox{45}{\makebox(0,0){\strut{}$  1 \times10^{4}$}}}%
      \csname LTb\endcsname%%
      \put(872,448){\makebox(0,0){\strut{}1749}}%
      \csname LTb\endcsname%%
      \put(1742,448){\makebox(0,0){\strut{}1779}}%
      \csname LTb\endcsname%%
      \put(2611,448){\makebox(0,0){\strut{}1809}}%
      \csname LTb\endcsname%%
      \put(3481,448){\makebox(0,0){\strut{}1839}}%
      \csname LTb\endcsname%%
      \put(4350,448){\makebox(0,0){\strut{}1869}}%
      \csname LTb\endcsname%%
      \put(5220,448){\makebox(0,0){\strut{}1899}}%
      \csname LTb\endcsname%%
      \put(6089,448){\makebox(0,0){\strut{}1930}}%
      \csname LTb\endcsname%%
      \put(6959,448){\makebox(0,0){\strut{}1960}}%
      \csname LTb\endcsname%%
      \put(7829,448){\makebox(0,0){\strut{}1990}}%
      \csname LTb\endcsname%%
      \put(8698,448){\makebox(0,0){\strut{}2020}}%
    }%
    \gplgaddtomacro\gplfronttext{%
      \csname LTb\endcsname%%
      \put(74,2559){\rotatebox{-270}{\makebox(0,0){\strut{}$ Y_n$}}}%
      \csname LTb\endcsname%%
      \put(4787,142){\makebox(0,0){\strut{}t [year]}}%
      \csname LTb\endcsname%%
      \put(2401,1039){\makebox(0,0)[r]{\strut{}815}}%
      \csname LTb\endcsname%%
      \put(2401,835){\makebox(0,0)[r]{\strut{}576}}%
      \csname LTb\endcsname%%
      \put(3490,1039){\makebox(0,0)[r]{\strut{}450}}%
      \csname LTb\endcsname%%
      \put(3490,835){\makebox(0,0)[r]{\strut{}400}}%
      \csname LTb\endcsname%%
      \put(4579,1039){\makebox(0,0)[r]{\strut{}311}}%
      \csname LTb\endcsname%%
      \put(4579,835){\makebox(0,0)[r]{\strut{}280}}%
      \csname LTb\endcsname%%
      \put(5668,1039){\makebox(0,0)[r]{\strut{}220}}%
      \csname LTb\endcsname%%
      \put(5668,835){\makebox(0,0)[r]{\strut{}132}}%
      \csname LTb\endcsname%%
      \put(6757,1039){\makebox(0,0)[r]{\strut{}60}}%
      \csname LTb\endcsname%%
      \put(6757,835){\makebox(0,0)[r]{\strut{}17}}%
      \csname LTb\endcsname%%
      \put(4787,4773){\makebox(0,0){\strut{}profile function for sun spots number}}%
    }%
    \gplbacktext
    \put(0,0){\includegraphics[width={453.00bp},height={255.00bp}]{../img/true/profilePlot}}%
    \gplfronttext
  \end{picture}%
\endgroup
}
			\end{block}
		\end{column}
	\end{columns}
\end{frame}

\begin{frame}{Sunspots}{DFA1}	
		\small{Vertical black lines at $\tau = 4,17, 132, 450 \text{ months}$.}
	\begin{columns}
		\begin{column}{.47\textwidth}
			\centering
			\begin{block}{\small{daily sunspot number}}
			\scalebox{.7}{% GNUPLOT: LaTeX picture with Postscript
\begingroup
  \makeatletter
  \providecommand\color[2][]{%
    \GenericError{(gnuplot) \space\space\space\@spaces}{%
      Package color not loaded in conjunction with
      terminal option `colourtext'%
    }{See the gnuplot documentation for explanation.%
    }{Either use 'blacktext' in gnuplot or load the package
      color.sty in LaTeX.}%
    \renewcommand\color[2][]{}%
  }%
  \providecommand\includegraphics[2][]{%
    \GenericError{(gnuplot) \space\space\space\@spaces}{%
      Package graphicx or graphics not loaded%
    }{See the gnuplot documentation for explanation.%
    }{The gnuplot epslatex terminal needs graphicx.sty or graphics.sty.}%
    \renewcommand\includegraphics[2][]{}%
  }%
  \providecommand\rotatebox[2]{#2}%
  \@ifundefined{ifGPcolor}{%
    \newif\ifGPcolor
    \GPcolortrue
  }{}%
  \@ifundefined{ifGPblacktext}{%
    \newif\ifGPblacktext
    \GPblacktexttrue
  }{}%
  % define a \g@addto@macro without @ in the name:
  \let\gplgaddtomacro\g@addto@macro
  % define empty templates for all commands taking text:
  \gdef\gplbacktext{}%
  \gdef\gplfronttext{}%
  \makeatother
  \ifGPblacktext
    % no textcolor at all
    \def\colorrgb#1{}%
    \def\colorgray#1{}%
  \else
    % gray or color?
    \ifGPcolor
      \def\colorrgb#1{\color[rgb]{#1}}%
      \def\colorgray#1{\color[gray]{#1}}%
      \expandafter\def\csname LTw\endcsname{\color{white}}%
      \expandafter\def\csname LTb\endcsname{\color{black}}%
      \expandafter\def\csname LTa\endcsname{\color{black}}%
      \expandafter\def\csname LT0\endcsname{\color[rgb]{1,0,0}}%
      \expandafter\def\csname LT1\endcsname{\color[rgb]{0,1,0}}%
      \expandafter\def\csname LT2\endcsname{\color[rgb]{0,0,1}}%
      \expandafter\def\csname LT3\endcsname{\color[rgb]{1,0,1}}%
      \expandafter\def\csname LT4\endcsname{\color[rgb]{0,1,1}}%
      \expandafter\def\csname LT5\endcsname{\color[rgb]{1,1,0}}%
      \expandafter\def\csname LT6\endcsname{\color[rgb]{0,0,0}}%
      \expandafter\def\csname LT7\endcsname{\color[rgb]{1,0.3,0}}%
      \expandafter\def\csname LT8\endcsname{\color[rgb]{0.5,0.5,0.5}}%
    \else
      % gray
      \def\colorrgb#1{\color{black}}%
      \def\colorgray#1{\color[gray]{#1}}%
      \expandafter\def\csname LTw\endcsname{\color{white}}%
      \expandafter\def\csname LTb\endcsname{\color{black}}%
      \expandafter\def\csname LTa\endcsname{\color{black}}%
      \expandafter\def\csname LT0\endcsname{\color{black}}%
      \expandafter\def\csname LT1\endcsname{\color{black}}%
      \expandafter\def\csname LT2\endcsname{\color{black}}%
      \expandafter\def\csname LT3\endcsname{\color{black}}%
      \expandafter\def\csname LT4\endcsname{\color{black}}%
      \expandafter\def\csname LT5\endcsname{\color{black}}%
      \expandafter\def\csname LT6\endcsname{\color{black}}%
      \expandafter\def\csname LT7\endcsname{\color{black}}%
      \expandafter\def\csname LT8\endcsname{\color{black}}%
    \fi
  \fi
    \setlength{\unitlength}{0.0500bp}%
    \ifx\gptboxheight\undefined%
      \newlength{\gptboxheight}%
      \newlength{\gptboxwidth}%
      \newsavebox{\gptboxtext}%
    \fi%
    \setlength{\fboxrule}{0.5pt}%
    \setlength{\fboxsep}{1pt}%
\begin{picture}(9060.00,5100.00)%
    \gplgaddtomacro\gplbacktext{%
      \csname LTb\endcsname%%
      \put(820,727){\makebox(0,0)[r]{\strut{}$ 10^{1}$}}%
      \csname LTb\endcsname%%
      \put(820,2122){\makebox(0,0)[r]{\strut{}$ 10^{2}$}}%
      \csname LTb\endcsname%%
      \put(820,3517){\makebox(0,0)[r]{\strut{}$ 10^{3}$}}%
      \csname LTb\endcsname%%
      \put(1058,448){\makebox(0,0){\strut{}4 months}}%
      \csname LTb\endcsname%%
      \put(2456,448){\makebox(0,0){\strut{}1 year}}%
      \csname LTb\endcsname%%
      \put(4219,448){\makebox(0,0){\strut{}4 years}}%
      \csname LTb\endcsname%%
      \put(5982,448){\makebox(0,0){\strut{}16 years}}%
      \csname LTb\endcsname%%
      \put(7380,448){\makebox(0,0){\strut{}48 years}}%
    }%
    \gplgaddtomacro\gplfronttext{%
      \csname LTb\endcsname%%
      \put(186,2559){\rotatebox{-270}{\makebox(0,0){\strut{}F(s)}}}%
      \csname LTb\endcsname%%
      \put(4817,142){\makebox(0,0){\strut{}time range}}%
      \csname LTb\endcsname%%
      \put(7838,1651){\makebox(0,0)[r]{\strut{}DFA}}%
      \csname LTb\endcsname%%
      \put(7838,1447){\makebox(0,0)[r]{\strut{}$\alpha$ = 0.750041}}%
      \csname LTb\endcsname%%
      \put(7838,1243){\makebox(0,0)[r]{\strut{}$\alpha$ = 0.326350}}%
      \csname LTb\endcsname%%
      \put(7838,1039){\makebox(0,0)[r]{\strut{}$\alpha$ = 1.649237}}%
      \csname LTb\endcsname%%
      \put(7838,835){\makebox(0,0)[r]{\strut{}$\alpha$ = 0.932191}}%
      \csname LTb\endcsname%%
      \put(4817,4773){\makebox(0,0){\strut{}DFA1 analysis for q=2}}%
    }%
    \gplbacktext
    \put(0,0){\includegraphics[width={453.00bp},height={255.00bp}]{../img/monthly/DFAPlot}}%
    \gplfronttext
  \end{picture}%
\endgroup
}
			\end{block}
		\end{column}
		\hfill
		\begin{column}{.47\textwidth}
			\centering
			\begin{block}{\small{monthly mean sunspot number}}
			\scalebox{.7}{% GNUPLOT: LaTeX picture with Postscript
\begingroup
  \makeatletter
  \providecommand\color[2][]{%
    \GenericError{(gnuplot) \space\space\space\@spaces}{%
      Package color not loaded in conjunction with
      terminal option `colourtext'%
    }{See the gnuplot documentation for explanation.%
    }{Either use 'blacktext' in gnuplot or load the package
      color.sty in LaTeX.}%
    \renewcommand\color[2][]{}%
  }%
  \providecommand\includegraphics[2][]{%
    \GenericError{(gnuplot) \space\space\space\@spaces}{%
      Package graphicx or graphics not loaded%
    }{See the gnuplot documentation for explanation.%
    }{The gnuplot epslatex terminal needs graphicx.sty or graphics.sty.}%
    \renewcommand\includegraphics[2][]{}%
  }%
  \providecommand\rotatebox[2]{#2}%
  \@ifundefined{ifGPcolor}{%
    \newif\ifGPcolor
    \GPcolortrue
  }{}%
  \@ifundefined{ifGPblacktext}{%
    \newif\ifGPblacktext
    \GPblacktexttrue
  }{}%
  % define a \g@addto@macro without @ in the name:
  \let\gplgaddtomacro\g@addto@macro
  % define empty templates for all commands taking text:
  \gdef\gplbacktext{}%
  \gdef\gplfronttext{}%
  \makeatother
  \ifGPblacktext
    % no textcolor at all
    \def\colorrgb#1{}%
    \def\colorgray#1{}%
  \else
    % gray or color?
    \ifGPcolor
      \def\colorrgb#1{\color[rgb]{#1}}%
      \def\colorgray#1{\color[gray]{#1}}%
      \expandafter\def\csname LTw\endcsname{\color{white}}%
      \expandafter\def\csname LTb\endcsname{\color{black}}%
      \expandafter\def\csname LTa\endcsname{\color{black}}%
      \expandafter\def\csname LT0\endcsname{\color[rgb]{1,0,0}}%
      \expandafter\def\csname LT1\endcsname{\color[rgb]{0,1,0}}%
      \expandafter\def\csname LT2\endcsname{\color[rgb]{0,0,1}}%
      \expandafter\def\csname LT3\endcsname{\color[rgb]{1,0,1}}%
      \expandafter\def\csname LT4\endcsname{\color[rgb]{0,1,1}}%
      \expandafter\def\csname LT5\endcsname{\color[rgb]{1,1,0}}%
      \expandafter\def\csname LT6\endcsname{\color[rgb]{0,0,0}}%
      \expandafter\def\csname LT7\endcsname{\color[rgb]{1,0.3,0}}%
      \expandafter\def\csname LT8\endcsname{\color[rgb]{0.5,0.5,0.5}}%
    \else
      % gray
      \def\colorrgb#1{\color{black}}%
      \def\colorgray#1{\color[gray]{#1}}%
      \expandafter\def\csname LTw\endcsname{\color{white}}%
      \expandafter\def\csname LTb\endcsname{\color{black}}%
      \expandafter\def\csname LTa\endcsname{\color{black}}%
      \expandafter\def\csname LT0\endcsname{\color{black}}%
      \expandafter\def\csname LT1\endcsname{\color{black}}%
      \expandafter\def\csname LT2\endcsname{\color{black}}%
      \expandafter\def\csname LT3\endcsname{\color{black}}%
      \expandafter\def\csname LT4\endcsname{\color{black}}%
      \expandafter\def\csname LT5\endcsname{\color{black}}%
      \expandafter\def\csname LT6\endcsname{\color{black}}%
      \expandafter\def\csname LT7\endcsname{\color{black}}%
      \expandafter\def\csname LT8\endcsname{\color{black}}%
    \fi
  \fi
    \setlength{\unitlength}{0.0500bp}%
    \ifx\gptboxheight\undefined%
      \newlength{\gptboxheight}%
      \newlength{\gptboxwidth}%
      \newsavebox{\gptboxtext}%
    \fi%
    \setlength{\fboxrule}{0.5pt}%
    \setlength{\fboxsep}{1pt}%
\begin{picture}(9060.00,5100.00)%
    \gplgaddtomacro\gplbacktext{%
      \csname LTb\endcsname%%
      \put(820,727){\makebox(0,0)[r]{\strut{}$ 10^{1}$}}%
      \csname LTb\endcsname%%
      \put(820,2122){\makebox(0,0)[r]{\strut{}$ 10^{2}$}}%
      \csname LTb\endcsname%%
      \put(820,3517){\makebox(0,0)[r]{\strut{}$ 10^{3}$}}%
      \csname LTb\endcsname%%
      \put(1058,448){\makebox(0,0){\strut{}4 months}}%
      \csname LTb\endcsname%%
      \put(2456,448){\makebox(0,0){\strut{}1 year}}%
      \csname LTb\endcsname%%
      \put(4219,448){\makebox(0,0){\strut{}4 years}}%
      \csname LTb\endcsname%%
      \put(5982,448){\makebox(0,0){\strut{}16 years}}%
      \csname LTb\endcsname%%
      \put(7380,448){\makebox(0,0){\strut{}48 years}}%
    }%
    \gplgaddtomacro\gplfronttext{%
      \csname LTb\endcsname%%
      \put(186,2559){\rotatebox{-270}{\makebox(0,0){\strut{}F(s)}}}%
      \csname LTb\endcsname%%
      \put(4817,142){\makebox(0,0){\strut{}time range}}%
      \csname LTb\endcsname%%
      \put(7838,1651){\makebox(0,0)[r]{\strut{}DFA}}%
      \csname LTb\endcsname%%
      \put(7838,1447){\makebox(0,0)[r]{\strut{}$\alpha$ = 0.750041}}%
      \csname LTb\endcsname%%
      \put(7838,1243){\makebox(0,0)[r]{\strut{}$\alpha$ = 0.326350}}%
      \csname LTb\endcsname%%
      \put(7838,1039){\makebox(0,0)[r]{\strut{}$\alpha$ = 1.649237}}%
      \csname LTb\endcsname%%
      \put(7838,835){\makebox(0,0)[r]{\strut{}$\alpha$ = 0.932191}}%
      \csname LTb\endcsname%%
      \put(4817,4773){\makebox(0,0){\strut{}DFA1 analysis for q=2}}%
    }%
    \gplbacktext
    \put(0,0){\includegraphics[width={453.00bp},height={255.00bp}]{../img/monthly/DFAPlot}}%
    \gplfronttext
  \end{picture}%
\endgroup
}
			\end{block}
		\end{column}
	\end{columns}
\end{frame}

\begin{frame}{Sunspots}{Profile fitted with polynomials of 7th order}
			\centering
			\scalebox{.7}{% GNUPLOT: LaTeX picture with Postscript
\begingroup
  \makeatletter
  \providecommand\color[2][]{%
    \GenericError{(gnuplot) \space\space\space\@spaces}{%
      Package color not loaded in conjunction with
      terminal option `colourtext'%
    }{See the gnuplot documentation for explanation.%
    }{Either use 'blacktext' in gnuplot or load the package
      color.sty in LaTeX.}%
    \renewcommand\color[2][]{}%
  }%
  \providecommand\includegraphics[2][]{%
    \GenericError{(gnuplot) \space\space\space\@spaces}{%
      Package graphicx or graphics not loaded%
    }{See the gnuplot documentation for explanation.%
    }{The gnuplot epslatex terminal needs graphicx.sty or graphics.sty.}%
    \renewcommand\includegraphics[2][]{}%
  }%
  \providecommand\rotatebox[2]{#2}%
  \@ifundefined{ifGPcolor}{%
    \newif\ifGPcolor
    \GPcolortrue
  }{}%
  \@ifundefined{ifGPblacktext}{%
    \newif\ifGPblacktext
    \GPblacktexttrue
  }{}%
  % define a \g@addto@macro without @ in the name:
  \let\gplgaddtomacro\g@addto@macro
  % define empty templates for all commands taking text:
  \gdef\gplbacktext{}%
  \gdef\gplfronttext{}%
  \makeatother
  \ifGPblacktext
    % no textcolor at all
    \def\colorrgb#1{}%
    \def\colorgray#1{}%
  \else
    % gray or color?
    \ifGPcolor
      \def\colorrgb#1{\color[rgb]{#1}}%
      \def\colorgray#1{\color[gray]{#1}}%
      \expandafter\def\csname LTw\endcsname{\color{white}}%
      \expandafter\def\csname LTb\endcsname{\color{black}}%
      \expandafter\def\csname LTa\endcsname{\color{black}}%
      \expandafter\def\csname LT0\endcsname{\color[rgb]{1,0,0}}%
      \expandafter\def\csname LT1\endcsname{\color[rgb]{0,1,0}}%
      \expandafter\def\csname LT2\endcsname{\color[rgb]{0,0,1}}%
      \expandafter\def\csname LT3\endcsname{\color[rgb]{1,0,1}}%
      \expandafter\def\csname LT4\endcsname{\color[rgb]{0,1,1}}%
      \expandafter\def\csname LT5\endcsname{\color[rgb]{1,1,0}}%
      \expandafter\def\csname LT6\endcsname{\color[rgb]{0,0,0}}%
      \expandafter\def\csname LT7\endcsname{\color[rgb]{1,0.3,0}}%
      \expandafter\def\csname LT8\endcsname{\color[rgb]{0.5,0.5,0.5}}%
    \else
      % gray
      \def\colorrgb#1{\color{black}}%
      \def\colorgray#1{\color[gray]{#1}}%
      \expandafter\def\csname LTw\endcsname{\color{white}}%
      \expandafter\def\csname LTb\endcsname{\color{black}}%
      \expandafter\def\csname LTa\endcsname{\color{black}}%
      \expandafter\def\csname LT0\endcsname{\color{black}}%
      \expandafter\def\csname LT1\endcsname{\color{black}}%
      \expandafter\def\csname LT2\endcsname{\color{black}}%
      \expandafter\def\csname LT3\endcsname{\color{black}}%
      \expandafter\def\csname LT4\endcsname{\color{black}}%
      \expandafter\def\csname LT5\endcsname{\color{black}}%
      \expandafter\def\csname LT6\endcsname{\color{black}}%
      \expandafter\def\csname LT7\endcsname{\color{black}}%
      \expandafter\def\csname LT8\endcsname{\color{black}}%
    \fi
  \fi
    \setlength{\unitlength}{0.0500bp}%
    \ifx\gptboxheight\undefined%
      \newlength{\gptboxheight}%
      \newlength{\gptboxwidth}%
      \newsavebox{\gptboxtext}%
    \fi%
    \setlength{\fboxrule}{0.5pt}%
    \setlength{\fboxsep}{1pt}%
\begin{picture}(9060.00,5100.00)%
    \gplgaddtomacro\gplbacktext{%
      \csname LTb\endcsname%%
      \put(520,836){\rotatebox{45}{\makebox(0,0){\strut{}$ -2 \times10^{4}$}}}%
      \csname LTb\endcsname%%
      \put(520,1547){\rotatebox{45}{\makebox(0,0){\strut{}$ -1 \times10^{4}$}}}%
      \csname LTb\endcsname%%
      \put(520,2259){\rotatebox{45}{\makebox(0,0){\strut{}$ -5 \times10^{3}$}}}%
      \csname LTb\endcsname%%
      \put(520,2970){\rotatebox{45}{\makebox(0,0){\strut{}$  0 \times10^{0}$}}}%
      \csname LTb\endcsname%%
      \put(520,3682){\rotatebox{45}{\makebox(0,0){\strut{}$  5 \times10^{3}$}}}%
      \csname LTb\endcsname%%
      \put(520,4394){\rotatebox{45}{\makebox(0,0){\strut{}$  1 \times10^{4}$}}}%
      \csname LTb\endcsname%%
      \put(872,448){\makebox(0,0){\strut{}1749}}%
      \csname LTb\endcsname%%
      \put(1742,448){\makebox(0,0){\strut{}1779}}%
      \csname LTb\endcsname%%
      \put(2611,448){\makebox(0,0){\strut{}1809}}%
      \csname LTb\endcsname%%
      \put(3481,448){\makebox(0,0){\strut{}1839}}%
      \csname LTb\endcsname%%
      \put(4350,448){\makebox(0,0){\strut{}1869}}%
      \csname LTb\endcsname%%
      \put(5220,448){\makebox(0,0){\strut{}1899}}%
      \csname LTb\endcsname%%
      \put(6089,448){\makebox(0,0){\strut{}1930}}%
      \csname LTb\endcsname%%
      \put(6959,448){\makebox(0,0){\strut{}1960}}%
      \csname LTb\endcsname%%
      \put(7829,448){\makebox(0,0){\strut{}1990}}%
      \csname LTb\endcsname%%
      \put(8698,448){\makebox(0,0){\strut{}2020}}%
    }%
    \gplgaddtomacro\gplfronttext{%
      \csname LTb\endcsname%%
      \put(74,2559){\rotatebox{-270}{\makebox(0,0){\strut{}$ Y_n$}}}%
      \csname LTb\endcsname%%
      \put(4787,142){\makebox(0,0){\strut{}t [year]}}%
      \csname LTb\endcsname%%
      \put(2401,1039){\makebox(0,0)[r]{\strut{}815}}%
      \csname LTb\endcsname%%
      \put(2401,835){\makebox(0,0)[r]{\strut{}576}}%
      \csname LTb\endcsname%%
      \put(3490,1039){\makebox(0,0)[r]{\strut{}450}}%
      \csname LTb\endcsname%%
      \put(3490,835){\makebox(0,0)[r]{\strut{}400}}%
      \csname LTb\endcsname%%
      \put(4579,1039){\makebox(0,0)[r]{\strut{}311}}%
      \csname LTb\endcsname%%
      \put(4579,835){\makebox(0,0)[r]{\strut{}280}}%
      \csname LTb\endcsname%%
      \put(5668,1039){\makebox(0,0)[r]{\strut{}220}}%
      \csname LTb\endcsname%%
      \put(5668,835){\makebox(0,0)[r]{\strut{}132}}%
      \csname LTb\endcsname%%
      \put(6757,1039){\makebox(0,0)[r]{\strut{}60}}%
      \csname LTb\endcsname%%
      \put(6757,835){\makebox(0,0)[r]{\strut{}17}}%
      \csname LTb\endcsname%%
      \put(4787,4773){\makebox(0,0){\strut{}profile function for sun spots number}}%
    }%
    \gplbacktext
    \put(0,0){\includegraphics[width={453.00bp},height={255.00bp}]{../img/true/profilePlot}}%
    \gplfronttext
  \end{picture}%
\endgroup
}
\end{frame}

\begin{frame}{Sunspots}{DFA$k$ comparison}
		\centering
		\scalebox{.7}{% GNUPLOT: LaTeX picture with Postscript
\begingroup
  \makeatletter
  \providecommand\color[2][]{%
    \GenericError{(gnuplot) \space\space\space\@spaces}{%
      Package color not loaded in conjunction with
      terminal option `colourtext'%
    }{See the gnuplot documentation for explanation.%
    }{Either use 'blacktext' in gnuplot or load the package
      color.sty in LaTeX.}%
    \renewcommand\color[2][]{}%
  }%
  \providecommand\includegraphics[2][]{%
    \GenericError{(gnuplot) \space\space\space\@spaces}{%
      Package graphicx or graphics not loaded%
    }{See the gnuplot documentation for explanation.%
    }{The gnuplot epslatex terminal needs graphicx.sty or graphics.sty.}%
    \renewcommand\includegraphics[2][]{}%
  }%
  \providecommand\rotatebox[2]{#2}%
  \@ifundefined{ifGPcolor}{%
    \newif\ifGPcolor
    \GPcolortrue
  }{}%
  \@ifundefined{ifGPblacktext}{%
    \newif\ifGPblacktext
    \GPblacktexttrue
  }{}%
  % define a \g@addto@macro without @ in the name:
  \let\gplgaddtomacro\g@addto@macro
  % define empty templates for all commands taking text:
  \gdef\gplbacktext{}%
  \gdef\gplfronttext{}%
  \makeatother
  \ifGPblacktext
    % no textcolor at all
    \def\colorrgb#1{}%
    \def\colorgray#1{}%
  \else
    % gray or color?
    \ifGPcolor
      \def\colorrgb#1{\color[rgb]{#1}}%
      \def\colorgray#1{\color[gray]{#1}}%
      \expandafter\def\csname LTw\endcsname{\color{white}}%
      \expandafter\def\csname LTb\endcsname{\color{black}}%
      \expandafter\def\csname LTa\endcsname{\color{black}}%
      \expandafter\def\csname LT0\endcsname{\color[rgb]{1,0,0}}%
      \expandafter\def\csname LT1\endcsname{\color[rgb]{0,1,0}}%
      \expandafter\def\csname LT2\endcsname{\color[rgb]{0,0,1}}%
      \expandafter\def\csname LT3\endcsname{\color[rgb]{1,0,1}}%
      \expandafter\def\csname LT4\endcsname{\color[rgb]{0,1,1}}%
      \expandafter\def\csname LT5\endcsname{\color[rgb]{1,1,0}}%
      \expandafter\def\csname LT6\endcsname{\color[rgb]{0,0,0}}%
      \expandafter\def\csname LT7\endcsname{\color[rgb]{1,0.3,0}}%
      \expandafter\def\csname LT8\endcsname{\color[rgb]{0.5,0.5,0.5}}%
    \else
      % gray
      \def\colorrgb#1{\color{black}}%
      \def\colorgray#1{\color[gray]{#1}}%
      \expandafter\def\csname LTw\endcsname{\color{white}}%
      \expandafter\def\csname LTb\endcsname{\color{black}}%
      \expandafter\def\csname LTa\endcsname{\color{black}}%
      \expandafter\def\csname LT0\endcsname{\color{black}}%
      \expandafter\def\csname LT1\endcsname{\color{black}}%
      \expandafter\def\csname LT2\endcsname{\color{black}}%
      \expandafter\def\csname LT3\endcsname{\color{black}}%
      \expandafter\def\csname LT4\endcsname{\color{black}}%
      \expandafter\def\csname LT5\endcsname{\color{black}}%
      \expandafter\def\csname LT6\endcsname{\color{black}}%
      \expandafter\def\csname LT7\endcsname{\color{black}}%
      \expandafter\def\csname LT8\endcsname{\color{black}}%
    \fi
  \fi
    \setlength{\unitlength}{0.0500bp}%
    \ifx\gptboxheight\undefined%
      \newlength{\gptboxheight}%
      \newlength{\gptboxwidth}%
      \newsavebox{\gptboxtext}%
    \fi%
    \setlength{\fboxrule}{0.5pt}%
    \setlength{\fboxsep}{1pt}%
\begin{picture}(9060.00,5100.00)%
    \gplgaddtomacro\gplbacktext{%
      \csname LTb\endcsname%%
      \put(820,652){\makebox(0,0)[r]{\strut{}$ 10^{0}$}}%
      \csname LTb\endcsname%%
      \put(820,1606){\makebox(0,0)[r]{\strut{}$ 10^{1}$}}%
      \csname LTb\endcsname%%
      \put(820,2560){\makebox(0,0)[r]{\strut{}$ 10^{2}$}}%
      \csname LTb\endcsname%%
      \put(820,3513){\makebox(0,0)[r]{\strut{}$ 10^{3}$}}%
      \csname LTb\endcsname%%
      \put(820,4467){\makebox(0,0)[r]{\strut{}$ 10^{4}$}}%
      \csname LTb\endcsname%%
      \put(2271,448){\makebox(0,0){\strut{}$ 10^{1}$}}%
      \csname LTb\endcsname%%
      \put(5637,448){\makebox(0,0){\strut{}$ 10^{2}$}}%
    }%
    \gplgaddtomacro\gplfronttext{%
      \csname LTb\endcsname%%
      \put(186,2559){\rotatebox{-270}{\makebox(0,0){\strut{}$ F^{k}_2 (\tau) $}}}%
      \csname LTb\endcsname%%
      \put(4817,142){\makebox(0,0){\strut{}$\tau$ [months]}}%
      \csname LTb\endcsname%%
      \put(4235,1447){\makebox(0,0)[r]{\strut{}DFA1}}%
      \csname LTb\endcsname%%
      \put(4235,1243){\makebox(0,0)[r]{\strut{}1.14}}%
      \csname LTb\endcsname%%
      \put(4235,1039){\makebox(0,0)[r]{\strut{}0.95}}%
      \csname LTb\endcsname%%
      \put(4235,835){\makebox(0,0)[r]{\strut{} }}%
      \csname LTb\endcsname%%
      \put(5436,1447){\makebox(0,0)[r]{\strut{}DFA2}}%
      \csname LTb\endcsname%%
      \put(5436,1243){\makebox(0,0)[r]{\strut{}1.20}}%
      \csname LTb\endcsname%%
      \put(5436,1039){\makebox(0,0)[r]{\strut{}0.77}}%
      \csname LTb\endcsname%%
      \put(5436,835){\makebox(0,0)[r]{\strut{} }}%
      \csname LTb\endcsname%%
      \put(6637,1447){\makebox(0,0)[r]{\strut{}DFA4}}%
      \csname LTb\endcsname%%
      \put(6637,1243){\makebox(0,0)[r]{\strut{}1.24}}%
      \csname LTb\endcsname%%
      \put(6637,1039){\makebox(0,0)[r]{\strut{}0.71}}%
      \csname LTb\endcsname%%
      \put(6637,835){\makebox(0,0)[r]{\strut{} }}%
      \csname LTb\endcsname%%
      \put(7838,1447){\makebox(0,0)[r]{\strut{}DFA7}}%
      \csname LTb\endcsname%%
      \put(7838,1243){\makebox(0,0)[r]{\strut{}1.23}}%
      \csname LTb\endcsname%%
      \put(7838,1039){\makebox(0,0)[r]{\strut{}0.71}}%
      \csname LTb\endcsname%%
      \put(7838,835){\makebox(0,0)[r]{\strut{}2.28}}%
      \csname LTb\endcsname%%
      \put(4817,4773){\makebox(0,0){\strut{}Comparison of DFAk analysis for q=2}}%
    }%
    \gplbacktext
    \put(0,0){\includegraphics[width={453.00bp},height={255.00bp}]{../img/true/DFAPlot}}%
    \gplfronttext
  \end{picture}%
\endgroup
}
\end{frame}

\begin{frame}{Sunspots}{Fourier-DFA}
	\begin{columns}
		\begin{column}{.47\textwidth}
				\centering
				\scalebox{.6}{\input{../sunSpots/img/fourier/f1profilePlot.tex}}
				\\\vspace{1ex}
				\tiny{profile without 50 Fourier-terms}
		\end{column}
		\hfill
		\begin{column}{.47\textwidth}
				\centering
				\scalebox{.6}{\input{../sunSpots/img/fourier/f2profilePlot.tex}}
				\\\vspace{1ex}
				\tiny{profile without 100 Fourier-terms}
		\end{column}
	\end{columns}
	\vfill
		\centering
		\scalebox{.5}{% GNUPLOT: LaTeX picture with Postscript
\begingroup
  \makeatletter
  \providecommand\color[2][]{%
    \GenericError{(gnuplot) \space\space\space\@spaces}{%
      Package color not loaded in conjunction with
      terminal option `colourtext'%
    }{See the gnuplot documentation for explanation.%
    }{Either use 'blacktext' in gnuplot or load the package
      color.sty in LaTeX.}%
    \renewcommand\color[2][]{}%
  }%
  \providecommand\includegraphics[2][]{%
    \GenericError{(gnuplot) \space\space\space\@spaces}{%
      Package graphicx or graphics not loaded%
    }{See the gnuplot documentation for explanation.%
    }{The gnuplot epslatex terminal needs graphicx.sty or graphics.sty.}%
    \renewcommand\includegraphics[2][]{}%
  }%
  \providecommand\rotatebox[2]{#2}%
  \@ifundefined{ifGPcolor}{%
    \newif\ifGPcolor
    \GPcolortrue
  }{}%
  \@ifundefined{ifGPblacktext}{%
    \newif\ifGPblacktext
    \GPblacktexttrue
  }{}%
  % define a \g@addto@macro without @ in the name:
  \let\gplgaddtomacro\g@addto@macro
  % define empty templates for all commands taking text:
  \gdef\gplbacktext{}%
  \gdef\gplfronttext{}%
  \makeatother
  \ifGPblacktext
    % no textcolor at all
    \def\colorrgb#1{}%
    \def\colorgray#1{}%
  \else
    % gray or color?
    \ifGPcolor
      \def\colorrgb#1{\color[rgb]{#1}}%
      \def\colorgray#1{\color[gray]{#1}}%
      \expandafter\def\csname LTw\endcsname{\color{white}}%
      \expandafter\def\csname LTb\endcsname{\color{black}}%
      \expandafter\def\csname LTa\endcsname{\color{black}}%
      \expandafter\def\csname LT0\endcsname{\color[rgb]{1,0,0}}%
      \expandafter\def\csname LT1\endcsname{\color[rgb]{0,1,0}}%
      \expandafter\def\csname LT2\endcsname{\color[rgb]{0,0,1}}%
      \expandafter\def\csname LT3\endcsname{\color[rgb]{1,0,1}}%
      \expandafter\def\csname LT4\endcsname{\color[rgb]{0,1,1}}%
      \expandafter\def\csname LT5\endcsname{\color[rgb]{1,1,0}}%
      \expandafter\def\csname LT6\endcsname{\color[rgb]{0,0,0}}%
      \expandafter\def\csname LT7\endcsname{\color[rgb]{1,0.3,0}}%
      \expandafter\def\csname LT8\endcsname{\color[rgb]{0.5,0.5,0.5}}%
    \else
      % gray
      \def\colorrgb#1{\color{black}}%
      \def\colorgray#1{\color[gray]{#1}}%
      \expandafter\def\csname LTw\endcsname{\color{white}}%
      \expandafter\def\csname LTb\endcsname{\color{black}}%
      \expandafter\def\csname LTa\endcsname{\color{black}}%
      \expandafter\def\csname LT0\endcsname{\color{black}}%
      \expandafter\def\csname LT1\endcsname{\color{black}}%
      \expandafter\def\csname LT2\endcsname{\color{black}}%
      \expandafter\def\csname LT3\endcsname{\color{black}}%
      \expandafter\def\csname LT4\endcsname{\color{black}}%
      \expandafter\def\csname LT5\endcsname{\color{black}}%
      \expandafter\def\csname LT6\endcsname{\color{black}}%
      \expandafter\def\csname LT7\endcsname{\color{black}}%
      \expandafter\def\csname LT8\endcsname{\color{black}}%
    \fi
  \fi
    \setlength{\unitlength}{0.0500bp}%
    \ifx\gptboxheight\undefined%
      \newlength{\gptboxheight}%
      \newlength{\gptboxwidth}%
      \newsavebox{\gptboxtext}%
    \fi%
    \setlength{\fboxrule}{0.5pt}%
    \setlength{\fboxsep}{1pt}%
\begin{picture}(9060.00,5100.00)%
    \gplgaddtomacro\gplbacktext{%
      \csname LTb\endcsname%%
      \put(820,727){\makebox(0,0)[r]{\strut{}$ 10^{1}$}}%
      \csname LTb\endcsname%%
      \put(820,2122){\makebox(0,0)[r]{\strut{}$ 10^{2}$}}%
      \csname LTb\endcsname%%
      \put(820,3517){\makebox(0,0)[r]{\strut{}$ 10^{3}$}}%
      \csname LTb\endcsname%%
      \put(1058,448){\makebox(0,0){\strut{}4 months}}%
      \csname LTb\endcsname%%
      \put(2456,448){\makebox(0,0){\strut{}1 year}}%
      \csname LTb\endcsname%%
      \put(4219,448){\makebox(0,0){\strut{}4 years}}%
      \csname LTb\endcsname%%
      \put(5982,448){\makebox(0,0){\strut{}16 years}}%
      \csname LTb\endcsname%%
      \put(7380,448){\makebox(0,0){\strut{}48 years}}%
    }%
    \gplgaddtomacro\gplfronttext{%
      \csname LTb\endcsname%%
      \put(186,2559){\rotatebox{-270}{\makebox(0,0){\strut{}F(s)}}}%
      \csname LTb\endcsname%%
      \put(4817,142){\makebox(0,0){\strut{}time range}}%
      \csname LTb\endcsname%%
      \put(7838,1651){\makebox(0,0)[r]{\strut{}DFA}}%
      \csname LTb\endcsname%%
      \put(7838,1447){\makebox(0,0)[r]{\strut{}$\alpha$ = 0.750041}}%
      \csname LTb\endcsname%%
      \put(7838,1243){\makebox(0,0)[r]{\strut{}$\alpha$ = 0.326350}}%
      \csname LTb\endcsname%%
      \put(7838,1039){\makebox(0,0)[r]{\strut{}$\alpha$ = 1.649237}}%
      \csname LTb\endcsname%%
      \put(7838,835){\makebox(0,0)[r]{\strut{}$\alpha$ = 0.932191}}%
      \csname LTb\endcsname%%
      \put(4817,4773){\makebox(0,0){\strut{}DFA1 analysis for q=2}}%
    }%
    \gplbacktext
    \put(0,0){\includegraphics[width={453.00bp},height={255.00bp}]{../img/monthly/DFAPlot}}%
    \gplfronttext
  \end{picture}%
\endgroup
}
\end{frame}

\end{document}
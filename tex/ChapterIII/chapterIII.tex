\graphicspath{{../sunSpots/img/}}
\section{Sunspots}
On the Sun's photosphere there is a strong magnetic field. However in some localized regions (called \textbf{sunspots}), the field is significant higher and the surface appear as spots darker than the surrounding areas.

\subsection{Raw data}
The SIDC, a department of the Royal Observatory of Belgium, provides, through one of its project, two databases of the sunspot number \cite{sidc}. \\
The first one provides daily measures from 01-01-1818 and the other one the monthly means from 01-01-1749. The relative plots can be found in \autoref{fig:ss}.
\begin{figure}[!h]
	\centering
	\begin{subfigure}{0.48\textwidth}
		% GNUPLOT: LaTeX picture with Postscript
\begingroup
  \makeatletter
  \providecommand\color[2][]{%
    \GenericError{(gnuplot) \space\space\space\@spaces}{%
      Package color not loaded in conjunction with
      terminal option `colourtext'%
    }{See the gnuplot documentation for explanation.%
    }{Either use 'blacktext' in gnuplot or load the package
      color.sty in LaTeX.}%
    \renewcommand\color[2][]{}%
  }%
  \providecommand\includegraphics[2][]{%
    \GenericError{(gnuplot) \space\space\space\@spaces}{%
      Package graphicx or graphics not loaded%
    }{See the gnuplot documentation for explanation.%
    }{The gnuplot epslatex terminal needs graphicx.sty or graphics.sty.}%
    \renewcommand\includegraphics[2][]{}%
  }%
  \providecommand\rotatebox[2]{#2}%
  \@ifundefined{ifGPcolor}{%
    \newif\ifGPcolor
    \GPcolortrue
  }{}%
  \@ifundefined{ifGPblacktext}{%
    \newif\ifGPblacktext
    \GPblacktexttrue
  }{}%
  % define a \g@addto@macro without @ in the name:
  \let\gplgaddtomacro\g@addto@macro
  % define empty templates for all commands taking text:
  \gdef\gplbacktext{}%
  \gdef\gplfronttext{}%
  \makeatother
  \ifGPblacktext
    % no textcolor at all
    \def\colorrgb#1{}%
    \def\colorgray#1{}%
  \else
    % gray or color?
    \ifGPcolor
      \def\colorrgb#1{\color[rgb]{#1}}%
      \def\colorgray#1{\color[gray]{#1}}%
      \expandafter\def\csname LTw\endcsname{\color{white}}%
      \expandafter\def\csname LTb\endcsname{\color{black}}%
      \expandafter\def\csname LTa\endcsname{\color{black}}%
      \expandafter\def\csname LT0\endcsname{\color[rgb]{1,0,0}}%
      \expandafter\def\csname LT1\endcsname{\color[rgb]{0,1,0}}%
      \expandafter\def\csname LT2\endcsname{\color[rgb]{0,0,1}}%
      \expandafter\def\csname LT3\endcsname{\color[rgb]{1,0,1}}%
      \expandafter\def\csname LT4\endcsname{\color[rgb]{0,1,1}}%
      \expandafter\def\csname LT5\endcsname{\color[rgb]{1,1,0}}%
      \expandafter\def\csname LT6\endcsname{\color[rgb]{0,0,0}}%
      \expandafter\def\csname LT7\endcsname{\color[rgb]{1,0.3,0}}%
      \expandafter\def\csname LT8\endcsname{\color[rgb]{0.5,0.5,0.5}}%
    \else
      % gray
      \def\colorrgb#1{\color{black}}%
      \def\colorgray#1{\color[gray]{#1}}%
      \expandafter\def\csname LTw\endcsname{\color{white}}%
      \expandafter\def\csname LTb\endcsname{\color{black}}%
      \expandafter\def\csname LTa\endcsname{\color{black}}%
      \expandafter\def\csname LT0\endcsname{\color{black}}%
      \expandafter\def\csname LT1\endcsname{\color{black}}%
      \expandafter\def\csname LT2\endcsname{\color{black}}%
      \expandafter\def\csname LT3\endcsname{\color{black}}%
      \expandafter\def\csname LT4\endcsname{\color{black}}%
      \expandafter\def\csname LT5\endcsname{\color{black}}%
      \expandafter\def\csname LT6\endcsname{\color{black}}%
      \expandafter\def\csname LT7\endcsname{\color{black}}%
      \expandafter\def\csname LT8\endcsname{\color{black}}%
    \fi
  \fi
    \setlength{\unitlength}{0.0500bp}%
    \ifx\gptboxheight\undefined%
      \newlength{\gptboxheight}%
      \newlength{\gptboxwidth}%
      \newsavebox{\gptboxtext}%
    \fi%
    \setlength{\fboxrule}{0.5pt}%
    \setlength{\fboxsep}{1pt}%
\begin{picture}(4520.00,2540.00)%
    \gplgaddtomacro\gplbacktext{%
      \csname LTb\endcsname%%
      \put(364,298){\makebox(0,0)[r]{\strut{}$0$}}%
      \csname LTb\endcsname%%
      \put(364,655){\makebox(0,0)[r]{\strut{}$100$}}%
      \csname LTb\endcsname%%
      \put(364,1011){\makebox(0,0)[r]{\strut{}$200$}}%
      \csname LTb\endcsname%%
      \put(364,1368){\makebox(0,0)[r]{\strut{}$300$}}%
      \csname LTb\endcsname%%
      \put(364,1725){\makebox(0,0)[r]{\strut{}$400$}}%
      \csname LTb\endcsname%%
      \put(364,2082){\makebox(0,0)[r]{\strut{}$500$}}%
      \csname LTb\endcsname%%
      \put(445,68){\rotatebox{45}{\makebox(0,0){\strut{}1818}}}%
      \csname LTb\endcsname%%
      \put(1001,68){\rotatebox{45}{\makebox(0,0){\strut{}1848}}}%
      \csname LTb\endcsname%%
      \put(1405,68){\rotatebox{45}{\makebox(0,0){\strut{}1869}}}%
      \csname LTb\endcsname%%
      \put(1803,68){\rotatebox{45}{\makebox(0,0){\strut{}1891}}}%
      \csname LTb\endcsname%%
      \put(2201,68){\rotatebox{45}{\makebox(0,0){\strut{}1912}}}%
      \csname LTb\endcsname%%
      \put(2598,68){\rotatebox{45}{\makebox(0,0){\strut{}1934}}}%
      \csname LTb\endcsname%%
      \put(2996,68){\rotatebox{45}{\makebox(0,0){\strut{}1955}}}%
      \csname LTb\endcsname%%
      \put(3394,68){\rotatebox{45}{\makebox(0,0){\strut{}1977}}}%
      \csname LTb\endcsname%%
      \put(3792,68){\rotatebox{45}{\makebox(0,0){\strut{}1998}}}%
      \csname LTb\endcsname%%
      \put(4190,68){\rotatebox{45}{\makebox(0,0){\strut{}2020}}}%
    }%
    \gplgaddtomacro\gplfronttext{%
    }%
    \gplbacktext
    \put(0,0){\includegraphics[width={226.00bp},height={127.00bp}]{../img/daily/dataPlot}}%
    \gplfronttext
  \end{picture}%
\endgroup

		\caption{Daily total sunspot number}\label{fig:dailyss}
	\end{subfigure}
	~
	\begin{subfigure}{0.48\textwidth}
		% GNUPLOT: LaTeX picture with Postscript
\begingroup
  \makeatletter
  \providecommand\color[2][]{%
    \GenericError{(gnuplot) \space\space\space\@spaces}{%
      Package color not loaded in conjunction with
      terminal option `colourtext'%
    }{See the gnuplot documentation for explanation.%
    }{Either use 'blacktext' in gnuplot or load the package
      color.sty in LaTeX.}%
    \renewcommand\color[2][]{}%
  }%
  \providecommand\includegraphics[2][]{%
    \GenericError{(gnuplot) \space\space\space\@spaces}{%
      Package graphicx or graphics not loaded%
    }{See the gnuplot documentation for explanation.%
    }{The gnuplot epslatex terminal needs graphicx.sty or graphics.sty.}%
    \renewcommand\includegraphics[2][]{}%
  }%
  \providecommand\rotatebox[2]{#2}%
  \@ifundefined{ifGPcolor}{%
    \newif\ifGPcolor
    \GPcolortrue
  }{}%
  \@ifundefined{ifGPblacktext}{%
    \newif\ifGPblacktext
    \GPblacktexttrue
  }{}%
  % define a \g@addto@macro without @ in the name:
  \let\gplgaddtomacro\g@addto@macro
  % define empty templates for all commands taking text:
  \gdef\gplbacktext{}%
  \gdef\gplfronttext{}%
  \makeatother
  \ifGPblacktext
    % no textcolor at all
    \def\colorrgb#1{}%
    \def\colorgray#1{}%
  \else
    % gray or color?
    \ifGPcolor
      \def\colorrgb#1{\color[rgb]{#1}}%
      \def\colorgray#1{\color[gray]{#1}}%
      \expandafter\def\csname LTw\endcsname{\color{white}}%
      \expandafter\def\csname LTb\endcsname{\color{black}}%
      \expandafter\def\csname LTa\endcsname{\color{black}}%
      \expandafter\def\csname LT0\endcsname{\color[rgb]{1,0,0}}%
      \expandafter\def\csname LT1\endcsname{\color[rgb]{0,1,0}}%
      \expandafter\def\csname LT2\endcsname{\color[rgb]{0,0,1}}%
      \expandafter\def\csname LT3\endcsname{\color[rgb]{1,0,1}}%
      \expandafter\def\csname LT4\endcsname{\color[rgb]{0,1,1}}%
      \expandafter\def\csname LT5\endcsname{\color[rgb]{1,1,0}}%
      \expandafter\def\csname LT6\endcsname{\color[rgb]{0,0,0}}%
      \expandafter\def\csname LT7\endcsname{\color[rgb]{1,0.3,0}}%
      \expandafter\def\csname LT8\endcsname{\color[rgb]{0.5,0.5,0.5}}%
    \else
      % gray
      \def\colorrgb#1{\color{black}}%
      \def\colorgray#1{\color[gray]{#1}}%
      \expandafter\def\csname LTw\endcsname{\color{white}}%
      \expandafter\def\csname LTb\endcsname{\color{black}}%
      \expandafter\def\csname LTa\endcsname{\color{black}}%
      \expandafter\def\csname LT0\endcsname{\color{black}}%
      \expandafter\def\csname LT1\endcsname{\color{black}}%
      \expandafter\def\csname LT2\endcsname{\color{black}}%
      \expandafter\def\csname LT3\endcsname{\color{black}}%
      \expandafter\def\csname LT4\endcsname{\color{black}}%
      \expandafter\def\csname LT5\endcsname{\color{black}}%
      \expandafter\def\csname LT6\endcsname{\color{black}}%
      \expandafter\def\csname LT7\endcsname{\color{black}}%
      \expandafter\def\csname LT8\endcsname{\color{black}}%
    \fi
  \fi
    \setlength{\unitlength}{0.0500bp}%
    \ifx\gptboxheight\undefined%
      \newlength{\gptboxheight}%
      \newlength{\gptboxwidth}%
      \newsavebox{\gptboxtext}%
    \fi%
    \setlength{\fboxrule}{0.5pt}%
    \setlength{\fboxsep}{1pt}%
\begin{picture}(9060.00,5100.00)%
    \gplgaddtomacro\gplbacktext{%
      \csname LTb\endcsname%%
      \put(708,652){\makebox(0,0)[r]{\strut{}$0$}}%
      \csname LTb\endcsname%%
      \put(708,1087){\makebox(0,0)[r]{\strut{}$50$}}%
      \csname LTb\endcsname%%
      \put(708,1523){\makebox(0,0)[r]{\strut{}$100$}}%
      \csname LTb\endcsname%%
      \put(708,1958){\makebox(0,0)[r]{\strut{}$150$}}%
      \csname LTb\endcsname%%
      \put(708,2394){\makebox(0,0)[r]{\strut{}$200$}}%
      \csname LTb\endcsname%%
      \put(708,2829){\makebox(0,0)[r]{\strut{}$250$}}%
      \csname LTb\endcsname%%
      \put(708,3265){\makebox(0,0)[r]{\strut{}$300$}}%
      \csname LTb\endcsname%%
      \put(708,3700){\makebox(0,0)[r]{\strut{}$350$}}%
      \csname LTb\endcsname%%
      \put(708,4136){\makebox(0,0)[r]{\strut{}$400$}}%
      \csname LTb\endcsname%%
      \put(820,448){\makebox(0,0){\strut{}1749}}%
      \csname LTb\endcsname%%
      \put(1695,448){\makebox(0,0){\strut{}1779}}%
      \csname LTb\endcsname%%
      \put(2571,448){\makebox(0,0){\strut{}1809}}%
      \csname LTb\endcsname%%
      \put(3446,448){\makebox(0,0){\strut{}1839}}%
      \csname LTb\endcsname%%
      \put(4321,448){\makebox(0,0){\strut{}1869}}%
      \csname LTb\endcsname%%
      \put(5197,448){\makebox(0,0){\strut{}1899}}%
      \csname LTb\endcsname%%
      \put(6072,448){\makebox(0,0){\strut{}1930}}%
      \csname LTb\endcsname%%
      \put(6947,448){\makebox(0,0){\strut{}1960}}%
      \csname LTb\endcsname%%
      \put(7823,448){\makebox(0,0){\strut{}1990}}%
      \csname LTb\endcsname%%
      \put(8698,448){\makebox(0,0){\strut{}2020}}%
    }%
    \gplgaddtomacro\gplfronttext{%
      \csname LTb\endcsname%%
      \put(186,2559){\rotatebox{-270}{\makebox(0,0){\strut{}Count}}}%
      \csname LTb\endcsname%%
      \put(4761,142){\makebox(0,0){\strut{}Time [year]}}%
      \csname LTb\endcsname%%
      \put(4761,4773){\makebox(0,0){\strut{}Monthly observed sun spots number}}%
    }%
    \gplbacktext
    \put(0,0){\includegraphics[width={453.00bp},height={255.00bp}]{../img/monthly/dataPlot}}%
    \gplfronttext
  \end{picture}%
\endgroup

		\caption{Monthly mean total sunspot number}\label{fig:monthlyss}
	\end{subfigure}
\caption{Time series $\{X_n\}$ of sunspot number, sampled at different frequency}\label{fig:ss}
\end{figure}

\subsection{Data Analysis}
DFAk, illustrated in \autoref{sec:dfa}, is applied to the sunspot databases. 

\subsubsection{Profile plot}
\autoref{fig:dailyprofile} and \autoref{fig:monthlyprofile} shows the profile $\{Y_n\}$ defined in \autoref{step1}. 

The profile is split in equal time windows $\tau$, as illustrated in \autoref{step2} and each division is fitted with a polynomial of $k$th order, as in \autoref{step3}.\\ 
As an example, in \autoref{fig:profilemain}, different time windows with their relative fit are plotted.
\begin{figure}[!h]
	\centering
	\begin{subfigure}{\textwidth}
		% GNUPLOT: LaTeX picture with Postscript
\begingroup
  \makeatletter
  \providecommand\color[2][]{%
    \GenericError{(gnuplot) \space\space\space\@spaces}{%
      Package color not loaded in conjunction with
      terminal option `colourtext'%
    }{See the gnuplot documentation for explanation.%
    }{Either use 'blacktext' in gnuplot or load the package
      color.sty in LaTeX.}%
    \renewcommand\color[2][]{}%
  }%
  \providecommand\includegraphics[2][]{%
    \GenericError{(gnuplot) \space\space\space\@spaces}{%
      Package graphicx or graphics not loaded%
    }{See the gnuplot documentation for explanation.%
    }{The gnuplot epslatex terminal needs graphicx.sty or graphics.sty.}%
    \renewcommand\includegraphics[2][]{}%
  }%
  \providecommand\rotatebox[2]{#2}%
  \@ifundefined{ifGPcolor}{%
    \newif\ifGPcolor
    \GPcolortrue
  }{}%
  \@ifundefined{ifGPblacktext}{%
    \newif\ifGPblacktext
    \GPblacktexttrue
  }{}%
  % define a \g@addto@macro without @ in the name:
  \let\gplgaddtomacro\g@addto@macro
  % define empty templates for all commands taking text:
  \gdef\gplbacktext{}%
  \gdef\gplfronttext{}%
  \makeatother
  \ifGPblacktext
    % no textcolor at all
    \def\colorrgb#1{}%
    \def\colorgray#1{}%
  \else
    % gray or color?
    \ifGPcolor
      \def\colorrgb#1{\color[rgb]{#1}}%
      \def\colorgray#1{\color[gray]{#1}}%
      \expandafter\def\csname LTw\endcsname{\color{white}}%
      \expandafter\def\csname LTb\endcsname{\color{black}}%
      \expandafter\def\csname LTa\endcsname{\color{black}}%
      \expandafter\def\csname LT0\endcsname{\color[rgb]{1,0,0}}%
      \expandafter\def\csname LT1\endcsname{\color[rgb]{0,1,0}}%
      \expandafter\def\csname LT2\endcsname{\color[rgb]{0,0,1}}%
      \expandafter\def\csname LT3\endcsname{\color[rgb]{1,0,1}}%
      \expandafter\def\csname LT4\endcsname{\color[rgb]{0,1,1}}%
      \expandafter\def\csname LT5\endcsname{\color[rgb]{1,1,0}}%
      \expandafter\def\csname LT6\endcsname{\color[rgb]{0,0,0}}%
      \expandafter\def\csname LT7\endcsname{\color[rgb]{1,0.3,0}}%
      \expandafter\def\csname LT8\endcsname{\color[rgb]{0.5,0.5,0.5}}%
    \else
      % gray
      \def\colorrgb#1{\color{black}}%
      \def\colorgray#1{\color[gray]{#1}}%
      \expandafter\def\csname LTw\endcsname{\color{white}}%
      \expandafter\def\csname LTb\endcsname{\color{black}}%
      \expandafter\def\csname LTa\endcsname{\color{black}}%
      \expandafter\def\csname LT0\endcsname{\color{black}}%
      \expandafter\def\csname LT1\endcsname{\color{black}}%
      \expandafter\def\csname LT2\endcsname{\color{black}}%
      \expandafter\def\csname LT3\endcsname{\color{black}}%
      \expandafter\def\csname LT4\endcsname{\color{black}}%
      \expandafter\def\csname LT5\endcsname{\color{black}}%
      \expandafter\def\csname LT6\endcsname{\color{black}}%
      \expandafter\def\csname LT7\endcsname{\color{black}}%
      \expandafter\def\csname LT8\endcsname{\color{black}}%
    \fi
  \fi
    \setlength{\unitlength}{0.0500bp}%
    \ifx\gptboxheight\undefined%
      \newlength{\gptboxheight}%
      \newlength{\gptboxwidth}%
      \newsavebox{\gptboxtext}%
    \fi%
    \setlength{\fboxrule}{0.5pt}%
    \setlength{\fboxsep}{1pt}%
\begin{picture}(9060.00,5100.00)%
    \gplgaddtomacro\gplbacktext{%
      \csname LTb\endcsname%%
      \put(520,836){\rotatebox{45}{\makebox(0,0){\strut{}$ -2 \times10^{4}$}}}%
      \csname LTb\endcsname%%
      \put(520,1547){\rotatebox{45}{\makebox(0,0){\strut{}$ -1 \times10^{4}$}}}%
      \csname LTb\endcsname%%
      \put(520,2259){\rotatebox{45}{\makebox(0,0){\strut{}$ -5 \times10^{3}$}}}%
      \csname LTb\endcsname%%
      \put(520,2970){\rotatebox{45}{\makebox(0,0){\strut{}$  0 \times10^{0}$}}}%
      \csname LTb\endcsname%%
      \put(520,3682){\rotatebox{45}{\makebox(0,0){\strut{}$  5 \times10^{3}$}}}%
      \csname LTb\endcsname%%
      \put(520,4394){\rotatebox{45}{\makebox(0,0){\strut{}$  1 \times10^{4}$}}}%
      \csname LTb\endcsname%%
      \put(872,448){\makebox(0,0){\strut{}1749}}%
      \csname LTb\endcsname%%
      \put(1742,448){\makebox(0,0){\strut{}1779}}%
      \csname LTb\endcsname%%
      \put(2611,448){\makebox(0,0){\strut{}1809}}%
      \csname LTb\endcsname%%
      \put(3481,448){\makebox(0,0){\strut{}1839}}%
      \csname LTb\endcsname%%
      \put(4350,448){\makebox(0,0){\strut{}1869}}%
      \csname LTb\endcsname%%
      \put(5220,448){\makebox(0,0){\strut{}1899}}%
      \csname LTb\endcsname%%
      \put(6089,448){\makebox(0,0){\strut{}1930}}%
      \csname LTb\endcsname%%
      \put(6959,448){\makebox(0,0){\strut{}1960}}%
      \csname LTb\endcsname%%
      \put(7829,448){\makebox(0,0){\strut{}1990}}%
      \csname LTb\endcsname%%
      \put(8698,448){\makebox(0,0){\strut{}2020}}%
    }%
    \gplgaddtomacro\gplfronttext{%
      \csname LTb\endcsname%%
      \put(74,2559){\rotatebox{-270}{\makebox(0,0){\strut{}$ Y_n$}}}%
      \csname LTb\endcsname%%
      \put(4787,142){\makebox(0,0){\strut{}t [year]}}%
      \csname LTb\endcsname%%
      \put(2401,1039){\makebox(0,0)[r]{\strut{}815}}%
      \csname LTb\endcsname%%
      \put(2401,835){\makebox(0,0)[r]{\strut{}576}}%
      \csname LTb\endcsname%%
      \put(3490,1039){\makebox(0,0)[r]{\strut{}450}}%
      \csname LTb\endcsname%%
      \put(3490,835){\makebox(0,0)[r]{\strut{}400}}%
      \csname LTb\endcsname%%
      \put(4579,1039){\makebox(0,0)[r]{\strut{}311}}%
      \csname LTb\endcsname%%
      \put(4579,835){\makebox(0,0)[r]{\strut{}280}}%
      \csname LTb\endcsname%%
      \put(5668,1039){\makebox(0,0)[r]{\strut{}220}}%
      \csname LTb\endcsname%%
      \put(5668,835){\makebox(0,0)[r]{\strut{}132}}%
      \csname LTb\endcsname%%
      \put(6757,1039){\makebox(0,0)[r]{\strut{}60}}%
      \csname LTb\endcsname%%
      \put(6757,835){\makebox(0,0)[r]{\strut{}17}}%
      \csname LTb\endcsname%%
      \put(4787,4773){\makebox(0,0){\strut{}profile function for sun spots number}}%
    }%
    \gplbacktext
    \put(0,0){\includegraphics[width={453.00bp},height={255.00bp}]{../img/true/profilePlot}}%
    \gplfronttext
  \end{picture}%
\endgroup

		\caption{Monthly mean sunspots \ref{fig:monthlyss} and example of fits. \\ 
		Coloured lines represent polynomial fits of 7-th order at different time scales, indicated in the legend and with months as unit.}\label{fig:profilemain}
	\end{subfigure}
	\begin{subfigure}{0.48\textwidth}
		% GNUPLOT: LaTeX picture with Postscript
\begingroup
  \makeatletter
  \providecommand\color[2][]{%
    \GenericError{(gnuplot) \space\space\space\@spaces}{%
      Package color not loaded in conjunction with
      terminal option `colourtext'%
    }{See the gnuplot documentation for explanation.%
    }{Either use 'blacktext' in gnuplot or load the package
      color.sty in LaTeX.}%
    \renewcommand\color[2][]{}%
  }%
  \providecommand\includegraphics[2][]{%
    \GenericError{(gnuplot) \space\space\space\@spaces}{%
      Package graphicx or graphics not loaded%
    }{See the gnuplot documentation for explanation.%
    }{The gnuplot epslatex terminal needs graphicx.sty or graphics.sty.}%
    \renewcommand\includegraphics[2][]{}%
  }%
  \providecommand\rotatebox[2]{#2}%
  \@ifundefined{ifGPcolor}{%
    \newif\ifGPcolor
    \GPcolortrue
  }{}%
  \@ifundefined{ifGPblacktext}{%
    \newif\ifGPblacktext
    \GPblacktexttrue
  }{}%
  % define a \g@addto@macro without @ in the name:
  \let\gplgaddtomacro\g@addto@macro
  % define empty templates for all commands taking text:
  \gdef\gplbacktext{}%
  \gdef\gplfronttext{}%
  \makeatother
  \ifGPblacktext
    % no textcolor at all
    \def\colorrgb#1{}%
    \def\colorgray#1{}%
  \else
    % gray or color?
    \ifGPcolor
      \def\colorrgb#1{\color[rgb]{#1}}%
      \def\colorgray#1{\color[gray]{#1}}%
      \expandafter\def\csname LTw\endcsname{\color{white}}%
      \expandafter\def\csname LTb\endcsname{\color{black}}%
      \expandafter\def\csname LTa\endcsname{\color{black}}%
      \expandafter\def\csname LT0\endcsname{\color[rgb]{1,0,0}}%
      \expandafter\def\csname LT1\endcsname{\color[rgb]{0,1,0}}%
      \expandafter\def\csname LT2\endcsname{\color[rgb]{0,0,1}}%
      \expandafter\def\csname LT3\endcsname{\color[rgb]{1,0,1}}%
      \expandafter\def\csname LT4\endcsname{\color[rgb]{0,1,1}}%
      \expandafter\def\csname LT5\endcsname{\color[rgb]{1,1,0}}%
      \expandafter\def\csname LT6\endcsname{\color[rgb]{0,0,0}}%
      \expandafter\def\csname LT7\endcsname{\color[rgb]{1,0.3,0}}%
      \expandafter\def\csname LT8\endcsname{\color[rgb]{0.5,0.5,0.5}}%
    \else
      % gray
      \def\colorrgb#1{\color{black}}%
      \def\colorgray#1{\color[gray]{#1}}%
      \expandafter\def\csname LTw\endcsname{\color{white}}%
      \expandafter\def\csname LTb\endcsname{\color{black}}%
      \expandafter\def\csname LTa\endcsname{\color{black}}%
      \expandafter\def\csname LT0\endcsname{\color{black}}%
      \expandafter\def\csname LT1\endcsname{\color{black}}%
      \expandafter\def\csname LT2\endcsname{\color{black}}%
      \expandafter\def\csname LT3\endcsname{\color{black}}%
      \expandafter\def\csname LT4\endcsname{\color{black}}%
      \expandafter\def\csname LT5\endcsname{\color{black}}%
      \expandafter\def\csname LT6\endcsname{\color{black}}%
      \expandafter\def\csname LT7\endcsname{\color{black}}%
      \expandafter\def\csname LT8\endcsname{\color{black}}%
    \fi
  \fi
    \setlength{\unitlength}{0.0500bp}%
    \ifx\gptboxheight\undefined%
      \newlength{\gptboxheight}%
      \newlength{\gptboxwidth}%
      \newsavebox{\gptboxtext}%
    \fi%
    \setlength{\fboxrule}{0.5pt}%
    \setlength{\fboxsep}{1pt}%
\begin{picture}(9060.00,5100.00)%
    \gplgaddtomacro\gplbacktext{%
      \csname LTb\endcsname%%
      \put(1064,849){\makebox(0,0)[r]{\strut{}$ -4 \times10^{5}$}}%
      \csname LTb\endcsname%%
      \put(1064,1395){\makebox(0,0)[r]{\strut{}$ -3 \times10^{5}$}}%
      \csname LTb\endcsname%%
      \put(1064,1941){\makebox(0,0)[r]{\strut{}$ -2 \times10^{5}$}}%
      \csname LTb\endcsname%%
      \put(1064,2486){\makebox(0,0)[r]{\strut{}$ -1 \times10^{5}$}}%
      \csname LTb\endcsname%%
      \put(1064,3032){\makebox(0,0)[r]{\strut{}$  0 \times10^{0}$}}%
      \csname LTb\endcsname%%
      \put(1064,3578){\makebox(0,0)[r]{\strut{}$  1 \times10^{5}$}}%
      \csname LTb\endcsname%%
      \put(1064,4124){\makebox(0,0)[r]{\strut{}$  2 \times10^{5}$}}%
      \csname LTb\endcsname%%
      \put(1176,448){\makebox(0,0){\strut{}1818}}%
      \csname LTb\endcsname%%
      \put(2294,448){\makebox(0,0){\strut{}1848}}%
      \csname LTb\endcsname%%
      \put(3105,448){\makebox(0,0){\strut{}1869}}%
      \csname LTb\endcsname%%
      \put(3905,448){\makebox(0,0){\strut{}1891}}%
      \csname LTb\endcsname%%
      \put(4705,448){\makebox(0,0){\strut{}1912}}%
      \csname LTb\endcsname%%
      \put(5504,448){\makebox(0,0){\strut{}1934}}%
      \csname LTb\endcsname%%
      \put(6304,448){\makebox(0,0){\strut{}1955}}%
      \csname LTb\endcsname%%
      \put(7104,448){\makebox(0,0){\strut{}1977}}%
      \csname LTb\endcsname%%
      \put(7903,448){\makebox(0,0){\strut{}1998}}%
      \csname LTb\endcsname%%
      \put(8703,448){\makebox(0,0){\strut{}2020}}%
    }%
    \gplgaddtomacro\gplfronttext{%
      \csname LTb\endcsname%%
      \put(4939,142){\makebox(0,0){\strut{}Time [year]}}%
      \csname LTb\endcsname%%
      \put(4939,4773){\makebox(0,0){\strut{}profile function for sun spots number}}%
    }%
    \gplbacktext
    \put(0,0){\includegraphics[width={453.00bp},height={255.00bp}]{../img/daily/profilePlot}}%
    \gplfronttext
  \end{picture}%
\endgroup

		\caption{Daily sunspots \ref{fig:dailyss}}\label{fig:dailyprofile}
	\end{subfigure}
	~
	\begin{subfigure}{0.48\textwidth}
		% GNUPLOT: LaTeX picture with Postscript
\begingroup
  \makeatletter
  \providecommand\color[2][]{%
    \GenericError{(gnuplot) \space\space\space\@spaces}{%
      Package color not loaded in conjunction with
      terminal option `colourtext'%
    }{See the gnuplot documentation for explanation.%
    }{Either use 'blacktext' in gnuplot or load the package
      color.sty in LaTeX.}%
    \renewcommand\color[2][]{}%
  }%
  \providecommand\includegraphics[2][]{%
    \GenericError{(gnuplot) \space\space\space\@spaces}{%
      Package graphicx or graphics not loaded%
    }{See the gnuplot documentation for explanation.%
    }{The gnuplot epslatex terminal needs graphicx.sty or graphics.sty.}%
    \renewcommand\includegraphics[2][]{}%
  }%
  \providecommand\rotatebox[2]{#2}%
  \@ifundefined{ifGPcolor}{%
    \newif\ifGPcolor
    \GPcolortrue
  }{}%
  \@ifundefined{ifGPblacktext}{%
    \newif\ifGPblacktext
    \GPblacktexttrue
  }{}%
  % define a \g@addto@macro without @ in the name:
  \let\gplgaddtomacro\g@addto@macro
  % define empty templates for all commands taking text:
  \gdef\gplbacktext{}%
  \gdef\gplfronttext{}%
  \makeatother
  \ifGPblacktext
    % no textcolor at all
    \def\colorrgb#1{}%
    \def\colorgray#1{}%
  \else
    % gray or color?
    \ifGPcolor
      \def\colorrgb#1{\color[rgb]{#1}}%
      \def\colorgray#1{\color[gray]{#1}}%
      \expandafter\def\csname LTw\endcsname{\color{white}}%
      \expandafter\def\csname LTb\endcsname{\color{black}}%
      \expandafter\def\csname LTa\endcsname{\color{black}}%
      \expandafter\def\csname LT0\endcsname{\color[rgb]{1,0,0}}%
      \expandafter\def\csname LT1\endcsname{\color[rgb]{0,1,0}}%
      \expandafter\def\csname LT2\endcsname{\color[rgb]{0,0,1}}%
      \expandafter\def\csname LT3\endcsname{\color[rgb]{1,0,1}}%
      \expandafter\def\csname LT4\endcsname{\color[rgb]{0,1,1}}%
      \expandafter\def\csname LT5\endcsname{\color[rgb]{1,1,0}}%
      \expandafter\def\csname LT6\endcsname{\color[rgb]{0,0,0}}%
      \expandafter\def\csname LT7\endcsname{\color[rgb]{1,0.3,0}}%
      \expandafter\def\csname LT8\endcsname{\color[rgb]{0.5,0.5,0.5}}%
    \else
      % gray
      \def\colorrgb#1{\color{black}}%
      \def\colorgray#1{\color[gray]{#1}}%
      \expandafter\def\csname LTw\endcsname{\color{white}}%
      \expandafter\def\csname LTb\endcsname{\color{black}}%
      \expandafter\def\csname LTa\endcsname{\color{black}}%
      \expandafter\def\csname LT0\endcsname{\color{black}}%
      \expandafter\def\csname LT1\endcsname{\color{black}}%
      \expandafter\def\csname LT2\endcsname{\color{black}}%
      \expandafter\def\csname LT3\endcsname{\color{black}}%
      \expandafter\def\csname LT4\endcsname{\color{black}}%
      \expandafter\def\csname LT5\endcsname{\color{black}}%
      \expandafter\def\csname LT6\endcsname{\color{black}}%
      \expandafter\def\csname LT7\endcsname{\color{black}}%
      \expandafter\def\csname LT8\endcsname{\color{black}}%
    \fi
  \fi
    \setlength{\unitlength}{0.0500bp}%
    \ifx\gptboxheight\undefined%
      \newlength{\gptboxheight}%
      \newlength{\gptboxwidth}%
      \newsavebox{\gptboxtext}%
    \fi%
    \setlength{\fboxrule}{0.5pt}%
    \setlength{\fboxsep}{1pt}%
\begin{picture}(9060.00,5100.00)%
    \gplgaddtomacro\gplbacktext{%
      \csname LTb\endcsname%%
      \put(1064,1376){\makebox(0,0)[r]{\strut{}$ -1 \times10^{4}$}}%
      \csname LTb\endcsname%%
      \put(1064,2130){\makebox(0,0)[r]{\strut{}$ -5 \times10^{3}$}}%
      \csname LTb\endcsname%%
      \put(1064,2883){\makebox(0,0)[r]{\strut{}$  0 \times10^{0}$}}%
      \csname LTb\endcsname%%
      \put(1064,3636){\makebox(0,0)[r]{\strut{}$  5 \times10^{3}$}}%
      \csname LTb\endcsname%%
      \put(1064,4389){\makebox(0,0)[r]{\strut{}$  1 \times10^{4}$}}%
      \csname LTb\endcsname%%
      \put(1176,448){\makebox(0,0){\strut{}1749}}%
      \csname LTb\endcsname%%
      \put(2012,448){\makebox(0,0){\strut{}1779}}%
      \csname LTb\endcsname%%
      \put(2848,448){\makebox(0,0){\strut{}1809}}%
      \csname LTb\endcsname%%
      \put(3683,448){\makebox(0,0){\strut{}1839}}%
      \csname LTb\endcsname%%
      \put(4519,448){\makebox(0,0){\strut{}1869}}%
      \csname LTb\endcsname%%
      \put(5355,448){\makebox(0,0){\strut{}1899}}%
      \csname LTb\endcsname%%
      \put(6191,448){\makebox(0,0){\strut{}1930}}%
      \csname LTb\endcsname%%
      \put(7027,448){\makebox(0,0){\strut{}1960}}%
      \csname LTb\endcsname%%
      \put(7863,448){\makebox(0,0){\strut{}1990}}%
      \csname LTb\endcsname%%
      \put(8698,448){\makebox(0,0){\strut{}2020}}%
    }%
    \gplgaddtomacro\gplfronttext{%
      \csname LTb\endcsname%%
      \put(4939,142){\makebox(0,0){\strut{}Time [year]}}%
      \csname LTb\endcsname%%
      \put(4939,4773){\makebox(0,0){\strut{}profile function for sun spots number}}%
    }%
    \gplbacktext
    \put(0,0){\includegraphics[width={453.00bp},height={255.00bp}]{../img/monthly/profilePlot}}%
    \gplfronttext
  \end{picture}%
\endgroup

		\caption{Monthly mean sunspots \ref{fig:monthlyss}}\label{fig:monthlyprofile}
	\end{subfigure}
	\caption{The profile $\{Y_n\}$, defined as \autoref{step1} and relative to data in \autoref{fig:ss}}\label{fig:profile}
\end{figure}

\subsubsection{Log-log plot}
As describe in \autoref{step4}, the total variance $F_2^{( k )}(\tau)$ is estimated for equally spaced $\log(\tau)$.

Firstly, neglecting the section with few points, DFA1 gives same results at the same $\tau$, independently from the frequency of sampling, as in \autoref{fig:dailyDFA1} and \autoref{fig:monthlyDFA1}.

Secondly, the plots shows that the the time series have different slopes, at different time scale. Ideally, however, if a process is self-affinal, the log-log plot of $F_2^{( k )}(\tau)$ should be a straight line. This could mean that the process has different behaviour at different time scale, or that the order of detrend $k$ is too low.

As it is illustrated in \autoref{fig:profilemain}, the order $k=7$ can't fit properly a time window $\tau > \SI{280}{months}$. Analogously, lower $k$ capability of fitting is limited. 

OLS can't fit the sunspots $Y_n$ with a polynomial of order 8 or bigger, because of the overflow of double in C program.

A comparison of DFA$k$ with different $k$ is plotted in \autoref{fig:DFAk}. 
The behaviour of the time series is the same in windows well detrended (dashed line).\\ 
The series in DFA7 shows two behaviour, and so is multi-fractal:
\begin{itemize}
	\item for $\tau \le \SI{132}{months}$, $\alpha(q) \in (0.5, 1)$ and so the process is a stationary long-range memory one.
	\item for $\SI{132}{months} < \tau \le \SI{280}{months}$, $\alpha(q) \in (2, 2.5)$ 
	\item for $\tau > \SI{280}{months}$, the series is not correct detrended.
\end{itemize}

In \autoref{sec:hdfa} nothing is said about process with $\alpha(q)>2$, and I have no idea on how to give an interpretation to this result. 

\begin{figure}[!h]
	\centering
	\begin{subfigure}{\textwidth}
		% GNUPLOT: LaTeX picture with Postscript
\begingroup
  \makeatletter
  \providecommand\color[2][]{%
    \GenericError{(gnuplot) \space\space\space\@spaces}{%
      Package color not loaded in conjunction with
      terminal option `colourtext'%
    }{See the gnuplot documentation for explanation.%
    }{Either use 'blacktext' in gnuplot or load the package
      color.sty in LaTeX.}%
    \renewcommand\color[2][]{}%
  }%
  \providecommand\includegraphics[2][]{%
    \GenericError{(gnuplot) \space\space\space\@spaces}{%
      Package graphicx or graphics not loaded%
    }{See the gnuplot documentation for explanation.%
    }{The gnuplot epslatex terminal needs graphicx.sty or graphics.sty.}%
    \renewcommand\includegraphics[2][]{}%
  }%
  \providecommand\rotatebox[2]{#2}%
  \@ifundefined{ifGPcolor}{%
    \newif\ifGPcolor
    \GPcolortrue
  }{}%
  \@ifundefined{ifGPblacktext}{%
    \newif\ifGPblacktext
    \GPblacktexttrue
  }{}%
  % define a \g@addto@macro without @ in the name:
  \let\gplgaddtomacro\g@addto@macro
  % define empty templates for all commands taking text:
  \gdef\gplbacktext{}%
  \gdef\gplfronttext{}%
  \makeatother
  \ifGPblacktext
    % no textcolor at all
    \def\colorrgb#1{}%
    \def\colorgray#1{}%
  \else
    % gray or color?
    \ifGPcolor
      \def\colorrgb#1{\color[rgb]{#1}}%
      \def\colorgray#1{\color[gray]{#1}}%
      \expandafter\def\csname LTw\endcsname{\color{white}}%
      \expandafter\def\csname LTb\endcsname{\color{black}}%
      \expandafter\def\csname LTa\endcsname{\color{black}}%
      \expandafter\def\csname LT0\endcsname{\color[rgb]{1,0,0}}%
      \expandafter\def\csname LT1\endcsname{\color[rgb]{0,1,0}}%
      \expandafter\def\csname LT2\endcsname{\color[rgb]{0,0,1}}%
      \expandafter\def\csname LT3\endcsname{\color[rgb]{1,0,1}}%
      \expandafter\def\csname LT4\endcsname{\color[rgb]{0,1,1}}%
      \expandafter\def\csname LT5\endcsname{\color[rgb]{1,1,0}}%
      \expandafter\def\csname LT6\endcsname{\color[rgb]{0,0,0}}%
      \expandafter\def\csname LT7\endcsname{\color[rgb]{1,0.3,0}}%
      \expandafter\def\csname LT8\endcsname{\color[rgb]{0.5,0.5,0.5}}%
    \else
      % gray
      \def\colorrgb#1{\color{black}}%
      \def\colorgray#1{\color[gray]{#1}}%
      \expandafter\def\csname LTw\endcsname{\color{white}}%
      \expandafter\def\csname LTb\endcsname{\color{black}}%
      \expandafter\def\csname LTa\endcsname{\color{black}}%
      \expandafter\def\csname LT0\endcsname{\color{black}}%
      \expandafter\def\csname LT1\endcsname{\color{black}}%
      \expandafter\def\csname LT2\endcsname{\color{black}}%
      \expandafter\def\csname LT3\endcsname{\color{black}}%
      \expandafter\def\csname LT4\endcsname{\color{black}}%
      \expandafter\def\csname LT5\endcsname{\color{black}}%
      \expandafter\def\csname LT6\endcsname{\color{black}}%
      \expandafter\def\csname LT7\endcsname{\color{black}}%
      \expandafter\def\csname LT8\endcsname{\color{black}}%
    \fi
  \fi
    \setlength{\unitlength}{0.0500bp}%
    \ifx\gptboxheight\undefined%
      \newlength{\gptboxheight}%
      \newlength{\gptboxwidth}%
      \newsavebox{\gptboxtext}%
    \fi%
    \setlength{\fboxrule}{0.5pt}%
    \setlength{\fboxsep}{1pt}%
\begin{picture}(9060.00,5100.00)%
    \gplgaddtomacro\gplbacktext{%
      \csname LTb\endcsname%%
      \put(820,652){\makebox(0,0)[r]{\strut{}$ 10^{0}$}}%
      \csname LTb\endcsname%%
      \put(820,1606){\makebox(0,0)[r]{\strut{}$ 10^{1}$}}%
      \csname LTb\endcsname%%
      \put(820,2560){\makebox(0,0)[r]{\strut{}$ 10^{2}$}}%
      \csname LTb\endcsname%%
      \put(820,3513){\makebox(0,0)[r]{\strut{}$ 10^{3}$}}%
      \csname LTb\endcsname%%
      \put(820,4467){\makebox(0,0)[r]{\strut{}$ 10^{4}$}}%
      \csname LTb\endcsname%%
      \put(932,448){\makebox(0,0){\strut{}4 months}}%
      \csname LTb\endcsname%%
      \put(2538,448){\makebox(0,0){\strut{}1 year}}%
      \csname LTb\endcsname%%
      \put(4564,448){\makebox(0,0){\strut{}4 years}}%
      \csname LTb\endcsname%%
      \put(6590,448){\makebox(0,0){\strut{}16 years}}%
      \csname LTb\endcsname%%
      \put(8196,448){\makebox(0,0){\strut{}48 years}}%
    }%
    \gplgaddtomacro\gplfronttext{%
      \csname LTb\endcsname%%
      \put(186,2559){\rotatebox{-270}{\makebox(0,0){\strut{}F(s)}}}%
      \csname LTb\endcsname%%
      \put(4817,142){\makebox(0,0){\strut{}time range}}%
      \csname LTb\endcsname%%
      \put(7838,2263){\makebox(0,0)[r]{\strut{}DFA1}}%
      \csname LTb\endcsname%%
      \put(7838,2059){\makebox(0,0)[r]{\strut{}DFA2}}%
      \csname LTb\endcsname%%
      \put(7838,1855){\makebox(0,0)[r]{\strut{}DFA4}}%
      \csname LTb\endcsname%%
      \put(7838,1651){\makebox(0,0)[r]{\strut{}DFA7}}%
      \csname LTb\endcsname%%
      \put(7838,1447){\makebox(0,0)[r]{\strut{}$\alpha$ = 1.14}}%
      \csname LTb\endcsname%%
      \put(7838,1243){\makebox(0,0)[r]{\strut{}$\alpha$ = 1.20}}%
      \csname LTb\endcsname%%
      \put(7838,1039){\makebox(0,0)[r]{\strut{}$\alpha$ = 1.24}}%
      \csname LTb\endcsname%%
      \put(7838,835){\makebox(0,0)[r]{\strut{}$\alpha$ = 1.23}}%
      \csname LTb\endcsname%%
      \put(4817,4773){\makebox(0,0){\strut{}DFAk for k=1, 2, 4, 7 analysis for q=2}}%
    }%
    \gplbacktext
    \put(0,0){\includegraphics[width={453.00bp},height={255.00bp}]{../img/true/DFAPlot}}%
    \gplfronttext
  \end{picture}%
\endgroup

		\caption{Monthly mean sunspots \ref{fig:monthlyss} DFAk}\label{fig:DFAk}
	\end{subfigure}

	\begin{subfigure}{0.48\textwidth}
		% GNUPLOT: LaTeX picture with Postscript
\begingroup
  \makeatletter
  \providecommand\color[2][]{%
    \GenericError{(gnuplot) \space\space\space\@spaces}{%
      Package color not loaded in conjunction with
      terminal option `colourtext'%
    }{See the gnuplot documentation for explanation.%
    }{Either use 'blacktext' in gnuplot or load the package
      color.sty in LaTeX.}%
    \renewcommand\color[2][]{}%
  }%
  \providecommand\includegraphics[2][]{%
    \GenericError{(gnuplot) \space\space\space\@spaces}{%
      Package graphicx or graphics not loaded%
    }{See the gnuplot documentation for explanation.%
    }{The gnuplot epslatex terminal needs graphicx.sty or graphics.sty.}%
    \renewcommand\includegraphics[2][]{}%
  }%
  \providecommand\rotatebox[2]{#2}%
  \@ifundefined{ifGPcolor}{%
    \newif\ifGPcolor
    \GPcolortrue
  }{}%
  \@ifundefined{ifGPblacktext}{%
    \newif\ifGPblacktext
    \GPblacktexttrue
  }{}%
  % define a \g@addto@macro without @ in the name:
  \let\gplgaddtomacro\g@addto@macro
  % define empty templates for all commands taking text:
  \gdef\gplbacktext{}%
  \gdef\gplfronttext{}%
  \makeatother
  \ifGPblacktext
    % no textcolor at all
    \def\colorrgb#1{}%
    \def\colorgray#1{}%
  \else
    % gray or color?
    \ifGPcolor
      \def\colorrgb#1{\color[rgb]{#1}}%
      \def\colorgray#1{\color[gray]{#1}}%
      \expandafter\def\csname LTw\endcsname{\color{white}}%
      \expandafter\def\csname LTb\endcsname{\color{black}}%
      \expandafter\def\csname LTa\endcsname{\color{black}}%
      \expandafter\def\csname LT0\endcsname{\color[rgb]{1,0,0}}%
      \expandafter\def\csname LT1\endcsname{\color[rgb]{0,1,0}}%
      \expandafter\def\csname LT2\endcsname{\color[rgb]{0,0,1}}%
      \expandafter\def\csname LT3\endcsname{\color[rgb]{1,0,1}}%
      \expandafter\def\csname LT4\endcsname{\color[rgb]{0,1,1}}%
      \expandafter\def\csname LT5\endcsname{\color[rgb]{1,1,0}}%
      \expandafter\def\csname LT6\endcsname{\color[rgb]{0,0,0}}%
      \expandafter\def\csname LT7\endcsname{\color[rgb]{1,0.3,0}}%
      \expandafter\def\csname LT8\endcsname{\color[rgb]{0.5,0.5,0.5}}%
    \else
      % gray
      \def\colorrgb#1{\color{black}}%
      \def\colorgray#1{\color[gray]{#1}}%
      \expandafter\def\csname LTw\endcsname{\color{white}}%
      \expandafter\def\csname LTb\endcsname{\color{black}}%
      \expandafter\def\csname LTa\endcsname{\color{black}}%
      \expandafter\def\csname LT0\endcsname{\color{black}}%
      \expandafter\def\csname LT1\endcsname{\color{black}}%
      \expandafter\def\csname LT2\endcsname{\color{black}}%
      \expandafter\def\csname LT3\endcsname{\color{black}}%
      \expandafter\def\csname LT4\endcsname{\color{black}}%
      \expandafter\def\csname LT5\endcsname{\color{black}}%
      \expandafter\def\csname LT6\endcsname{\color{black}}%
      \expandafter\def\csname LT7\endcsname{\color{black}}%
      \expandafter\def\csname LT8\endcsname{\color{black}}%
    \fi
  \fi
    \setlength{\unitlength}{0.0500bp}%
    \ifx\gptboxheight\undefined%
      \newlength{\gptboxheight}%
      \newlength{\gptboxwidth}%
      \newsavebox{\gptboxtext}%
    \fi%
    \setlength{\fboxrule}{0.5pt}%
    \setlength{\fboxsep}{1pt}%
\begin{picture}(9060.00,5100.00)%
    \gplgaddtomacro\gplbacktext{%
      \csname LTb\endcsname%%
      \put(820,1697){\makebox(0,0)[r]{\strut{}$ 10^{3}$}}%
      \csname LTb\endcsname%%
      \put(820,3270){\makebox(0,0)[r]{\strut{}$ 10^{4}$}}%
      \csname LTb\endcsname%%
      \put(1602,448){\makebox(0,0){\strut{}4 months}}%
      \csname LTb\endcsname%%
      \put(3138,448){\makebox(0,0){\strut{}1 years}}%
      \csname LTb\endcsname%%
      \put(5052,448){\makebox(0,0){\strut{}4 years}}%
      \csname LTb\endcsname%%
      \put(6966,448){\makebox(0,0){\strut{}16 years}}%
      \csname LTb\endcsname%%
      \put(8483,448){\makebox(0,0){\strut{}48 years}}%
    }%
    \gplgaddtomacro\gplfronttext{%
      \csname LTb\endcsname%%
      \put(186,2559){\rotatebox{-270}{\makebox(0,0){\strut{}F(s)}}}%
      \csname LTb\endcsname%%
      \put(4744,142){\makebox(0,0){\strut{}time range}}%
      \csname LTb\endcsname%%
      \put(7691,1651){\makebox(0,0)[r]{\strut{}DFA}}%
      \csname LTb\endcsname%%
      \put(7691,1447){\makebox(0,0)[r]{\strut{}$\alpha$ = 0.515683}}%
      \csname LTb\endcsname%%
      \put(7691,1243){\makebox(0,0)[r]{\strut{}$\alpha$ = 0.263930}}%
      \csname LTb\endcsname%%
      \put(7691,1039){\makebox(0,0)[r]{\strut{}$\alpha$ = 1.602106}}%
      \csname LTb\endcsname%%
      \put(7691,835){\makebox(0,0)[r]{\strut{}$\alpha$ = 0.924212}}%
      \csname LTb\endcsname%%
      \put(4744,4773){\makebox(0,0){\strut{}DFA1 analysis for q=2}}%
    }%
    \gplbacktext
    \put(0,0){\includegraphics[width={453.00bp},height={255.00bp}]{../img/daily/DFAPlot}}%
    \gplfronttext
  \end{picture}%
\endgroup

		\caption{Daily sunspot \ref{fig:dailyss} DFA1}\label{fig:dailyDFA1}
	\end{subfigure}
	~
	\begin{subfigure}{0.48\textwidth}
		% GNUPLOT: LaTeX picture with Postscript
\begingroup
  \makeatletter
  \providecommand\color[2][]{%
    \GenericError{(gnuplot) \space\space\space\@spaces}{%
      Package color not loaded in conjunction with
      terminal option `colourtext'%
    }{See the gnuplot documentation for explanation.%
    }{Either use 'blacktext' in gnuplot or load the package
      color.sty in LaTeX.}%
    \renewcommand\color[2][]{}%
  }%
  \providecommand\includegraphics[2][]{%
    \GenericError{(gnuplot) \space\space\space\@spaces}{%
      Package graphicx or graphics not loaded%
    }{See the gnuplot documentation for explanation.%
    }{The gnuplot epslatex terminal needs graphicx.sty or graphics.sty.}%
    \renewcommand\includegraphics[2][]{}%
  }%
  \providecommand\rotatebox[2]{#2}%
  \@ifundefined{ifGPcolor}{%
    \newif\ifGPcolor
    \GPcolortrue
  }{}%
  \@ifundefined{ifGPblacktext}{%
    \newif\ifGPblacktext
    \GPblacktexttrue
  }{}%
  % define a \g@addto@macro without @ in the name:
  \let\gplgaddtomacro\g@addto@macro
  % define empty templates for all commands taking text:
  \gdef\gplbacktext{}%
  \gdef\gplfronttext{}%
  \makeatother
  \ifGPblacktext
    % no textcolor at all
    \def\colorrgb#1{}%
    \def\colorgray#1{}%
  \else
    % gray or color?
    \ifGPcolor
      \def\colorrgb#1{\color[rgb]{#1}}%
      \def\colorgray#1{\color[gray]{#1}}%
      \expandafter\def\csname LTw\endcsname{\color{white}}%
      \expandafter\def\csname LTb\endcsname{\color{black}}%
      \expandafter\def\csname LTa\endcsname{\color{black}}%
      \expandafter\def\csname LT0\endcsname{\color[rgb]{1,0,0}}%
      \expandafter\def\csname LT1\endcsname{\color[rgb]{0,1,0}}%
      \expandafter\def\csname LT2\endcsname{\color[rgb]{0,0,1}}%
      \expandafter\def\csname LT3\endcsname{\color[rgb]{1,0,1}}%
      \expandafter\def\csname LT4\endcsname{\color[rgb]{0,1,1}}%
      \expandafter\def\csname LT5\endcsname{\color[rgb]{1,1,0}}%
      \expandafter\def\csname LT6\endcsname{\color[rgb]{0,0,0}}%
      \expandafter\def\csname LT7\endcsname{\color[rgb]{1,0.3,0}}%
      \expandafter\def\csname LT8\endcsname{\color[rgb]{0.5,0.5,0.5}}%
    \else
      % gray
      \def\colorrgb#1{\color{black}}%
      \def\colorgray#1{\color[gray]{#1}}%
      \expandafter\def\csname LTw\endcsname{\color{white}}%
      \expandafter\def\csname LTb\endcsname{\color{black}}%
      \expandafter\def\csname LTa\endcsname{\color{black}}%
      \expandafter\def\csname LT0\endcsname{\color{black}}%
      \expandafter\def\csname LT1\endcsname{\color{black}}%
      \expandafter\def\csname LT2\endcsname{\color{black}}%
      \expandafter\def\csname LT3\endcsname{\color{black}}%
      \expandafter\def\csname LT4\endcsname{\color{black}}%
      \expandafter\def\csname LT5\endcsname{\color{black}}%
      \expandafter\def\csname LT6\endcsname{\color{black}}%
      \expandafter\def\csname LT7\endcsname{\color{black}}%
      \expandafter\def\csname LT8\endcsname{\color{black}}%
    \fi
  \fi
    \setlength{\unitlength}{0.0500bp}%
    \ifx\gptboxheight\undefined%
      \newlength{\gptboxheight}%
      \newlength{\gptboxwidth}%
      \newsavebox{\gptboxtext}%
    \fi%
    \setlength{\fboxrule}{0.5pt}%
    \setlength{\fboxsep}{1pt}%
\begin{picture}(4520.00,2540.00)%
    \gplgaddtomacro\gplbacktext{%
      \csname LTb\endcsname%%
      \put(445,1102){\makebox(0,0)[r]{\strut{}$ 10^{2}$}}%
      \csname LTb\endcsname%%
      \put(445,1950){\makebox(0,0)[r]{\strut{}$ 10^{3}$}}%
      \csname LTb\endcsname%%
      \put(1169,149){\makebox(0,0){\strut{}$ 10^{1}$}}%
      \csname LTb\endcsname%%
      \put(2784,149){\makebox(0,0){\strut{}$ 10^{2}$}}%
    }%
    \gplgaddtomacro\gplfronttext{%
      \csname LTb\endcsname%%
      \put(3630,1137){\makebox(0,0)[r]{\strut{}0.88}}%
      \csname LTb\endcsname%%
      \put(3630,914){\makebox(0,0)[r]{\strut{}0.37}}%
      \csname LTb\endcsname%%
      \put(3630,691){\makebox(0,0)[r]{\strut{}1.59}}%
      \csname LTb\endcsname%%
      \put(3630,468){\makebox(0,0)[r]{\strut{}0.95}}%
    }%
    \gplbacktext
    \put(0,0){\includegraphics[width={226.00bp},height={127.00bp}]{../img/monthly/DFAPlot}}%
    \gplfronttext
  \end{picture}%
\endgroup

		\caption{Monthly mean sunspots \ref{fig:monthlyss} DFA1}\label{fig:monthlyDFA1}
	\end{subfigure}
	\caption{Detrended fluctuation analysis\\
	The estimation of the $\alpha(2)$ parameter is reported in the legend of graphs.\\
	In \autoref{fig:dailyDFA1} the unit of $\tau$ is [days] and in \autoref{fig:monthlyDFA1} is [months]. The vertical black lines are placed at same $\tau = 4,17, 132, 450 \text{ months}$.}\label{fig:DFA}
\end{figure}

%\begin{figure}[!h]
%	\centering
%	\begin{subfigure}{\textwidth}
%		% GNUPLOT: LaTeX picture with Postscript
\begingroup
  \makeatletter
  \providecommand\color[2][]{%
    \GenericError{(gnuplot) \space\space\space\@spaces}{%
      Package color not loaded in conjunction with
      terminal option `colourtext'%
    }{See the gnuplot documentation for explanation.%
    }{Either use 'blacktext' in gnuplot or load the package
      color.sty in LaTeX.}%
    \renewcommand\color[2][]{}%
  }%
  \providecommand\includegraphics[2][]{%
    \GenericError{(gnuplot) \space\space\space\@spaces}{%
      Package graphicx or graphics not loaded%
    }{See the gnuplot documentation for explanation.%
    }{The gnuplot epslatex terminal needs graphicx.sty or graphics.sty.}%
    \renewcommand\includegraphics[2][]{}%
  }%
  \providecommand\rotatebox[2]{#2}%
  \@ifundefined{ifGPcolor}{%
    \newif\ifGPcolor
    \GPcolortrue
  }{}%
  \@ifundefined{ifGPblacktext}{%
    \newif\ifGPblacktext
    \GPblacktexttrue
  }{}%
  % define a \g@addto@macro without @ in the name:
  \let\gplgaddtomacro\g@addto@macro
  % define empty templates for all commands taking text:
  \gdef\gplbacktext{}%
  \gdef\gplfronttext{}%
  \makeatother
  \ifGPblacktext
    % no textcolor at all
    \def\colorrgb#1{}%
    \def\colorgray#1{}%
  \else
    % gray or color?
    \ifGPcolor
      \def\colorrgb#1{\color[rgb]{#1}}%
      \def\colorgray#1{\color[gray]{#1}}%
      \expandafter\def\csname LTw\endcsname{\color{white}}%
      \expandafter\def\csname LTb\endcsname{\color{black}}%
      \expandafter\def\csname LTa\endcsname{\color{black}}%
      \expandafter\def\csname LT0\endcsname{\color[rgb]{1,0,0}}%
      \expandafter\def\csname LT1\endcsname{\color[rgb]{0,1,0}}%
      \expandafter\def\csname LT2\endcsname{\color[rgb]{0,0,1}}%
      \expandafter\def\csname LT3\endcsname{\color[rgb]{1,0,1}}%
      \expandafter\def\csname LT4\endcsname{\color[rgb]{0,1,1}}%
      \expandafter\def\csname LT5\endcsname{\color[rgb]{1,1,0}}%
      \expandafter\def\csname LT6\endcsname{\color[rgb]{0,0,0}}%
      \expandafter\def\csname LT7\endcsname{\color[rgb]{1,0.3,0}}%
      \expandafter\def\csname LT8\endcsname{\color[rgb]{0.5,0.5,0.5}}%
    \else
      % gray
      \def\colorrgb#1{\color{black}}%
      \def\colorgray#1{\color[gray]{#1}}%
      \expandafter\def\csname LTw\endcsname{\color{white}}%
      \expandafter\def\csname LTb\endcsname{\color{black}}%
      \expandafter\def\csname LTa\endcsname{\color{black}}%
      \expandafter\def\csname LT0\endcsname{\color{black}}%
      \expandafter\def\csname LT1\endcsname{\color{black}}%
      \expandafter\def\csname LT2\endcsname{\color{black}}%
      \expandafter\def\csname LT3\endcsname{\color{black}}%
      \expandafter\def\csname LT4\endcsname{\color{black}}%
      \expandafter\def\csname LT5\endcsname{\color{black}}%
      \expandafter\def\csname LT6\endcsname{\color{black}}%
      \expandafter\def\csname LT7\endcsname{\color{black}}%
      \expandafter\def\csname LT8\endcsname{\color{black}}%
    \fi
  \fi
    \setlength{\unitlength}{0.0500bp}%
    \ifx\gptboxheight\undefined%
      \newlength{\gptboxheight}%
      \newlength{\gptboxwidth}%
      \newsavebox{\gptboxtext}%
    \fi%
    \setlength{\fboxrule}{0.5pt}%
    \setlength{\fboxsep}{1pt}%
\begin{picture}(4520.00,2540.00)%
    \gplgaddtomacro\gplbacktext{%
      \csname LTb\endcsname%%
      \put(364,298){\makebox(0,0)[r]{\strut{}$0$}}%
      \csname LTb\endcsname%%
      \put(364,535){\makebox(0,0)[r]{\strut{}$50$}}%
      \csname LTb\endcsname%%
      \put(364,771){\makebox(0,0)[r]{\strut{}$100$}}%
      \csname LTb\endcsname%%
      \put(364,1008){\makebox(0,0)[r]{\strut{}$150$}}%
      \csname LTb\endcsname%%
      \put(364,1244){\makebox(0,0)[r]{\strut{}$200$}}%
      \csname LTb\endcsname%%
      \put(364,1481){\makebox(0,0)[r]{\strut{}$250$}}%
      \csname LTb\endcsname%%
      \put(364,1717){\makebox(0,0)[r]{\strut{}$300$}}%
      \csname LTb\endcsname%%
      \put(364,1954){\makebox(0,0)[r]{\strut{}$350$}}%
      \csname LTb\endcsname%%
      \put(364,2190){\makebox(0,0)[r]{\strut{}$400$}}%
      \csname LTb\endcsname%%
      \put(445,68){\rotatebox{45}{\makebox(0,0){\strut{}1749}}}%
      \csname LTb\endcsname%%
      \put(861,68){\rotatebox{45}{\makebox(0,0){\strut{}1779}}}%
      \csname LTb\endcsname%%
      \put(1277,68){\rotatebox{45}{\makebox(0,0){\strut{}1809}}}%
      \csname LTb\endcsname%%
      \put(1693,68){\rotatebox{45}{\makebox(0,0){\strut{}1839}}}%
      \csname LTb\endcsname%%
      \put(2109,68){\rotatebox{45}{\makebox(0,0){\strut{}1869}}}%
      \csname LTb\endcsname%%
      \put(2525,68){\rotatebox{45}{\makebox(0,0){\strut{}1899}}}%
      \csname LTb\endcsname%%
      \put(2941,68){\rotatebox{45}{\makebox(0,0){\strut{}1930}}}%
      \csname LTb\endcsname%%
      \put(3358,68){\rotatebox{45}{\makebox(0,0){\strut{}1960}}}%
      \csname LTb\endcsname%%
      \put(3774,68){\rotatebox{45}{\makebox(0,0){\strut{}1990}}}%
      \csname LTb\endcsname%%
      \put(4190,68){\rotatebox{45}{\makebox(0,0){\strut{}2020}}}%
    }%
    \gplgaddtomacro\gplfronttext{%
    }%
    \gplbacktext
    \put(0,0){\includegraphics[width={226.00bp},height={127.00bp}]{../img/fourier/dataPlot}}%
    \gplfronttext
  \end{picture}%
\endgroup

%	\end{subfigure}
%	
%	\begin{subfigure}{0.48\textwidth}
%		% GNUPLOT: LaTeX picture with Postscript
\begingroup
  \makeatletter
  \providecommand\color[2][]{%
    \GenericError{(gnuplot) \space\space\space\@spaces}{%
      Package color not loaded in conjunction with
      terminal option `colourtext'%
    }{See the gnuplot documentation for explanation.%
    }{Either use 'blacktext' in gnuplot or load the package
      color.sty in LaTeX.}%
    \renewcommand\color[2][]{}%
  }%
  \providecommand\includegraphics[2][]{%
    \GenericError{(gnuplot) \space\space\space\@spaces}{%
      Package graphicx or graphics not loaded%
    }{See the gnuplot documentation for explanation.%
    }{The gnuplot epslatex terminal needs graphicx.sty or graphics.sty.}%
    \renewcommand\includegraphics[2][]{}%
  }%
  \providecommand\rotatebox[2]{#2}%
  \@ifundefined{ifGPcolor}{%
    \newif\ifGPcolor
    \GPcolortrue
  }{}%
  \@ifundefined{ifGPblacktext}{%
    \newif\ifGPblacktext
    \GPblacktexttrue
  }{}%
  % define a \g@addto@macro without @ in the name:
  \let\gplgaddtomacro\g@addto@macro
  % define empty templates for all commands taking text:
  \gdef\gplbacktext{}%
  \gdef\gplfronttext{}%
  \makeatother
  \ifGPblacktext
    % no textcolor at all
    \def\colorrgb#1{}%
    \def\colorgray#1{}%
  \else
    % gray or color?
    \ifGPcolor
      \def\colorrgb#1{\color[rgb]{#1}}%
      \def\colorgray#1{\color[gray]{#1}}%
      \expandafter\def\csname LTw\endcsname{\color{white}}%
      \expandafter\def\csname LTb\endcsname{\color{black}}%
      \expandafter\def\csname LTa\endcsname{\color{black}}%
      \expandafter\def\csname LT0\endcsname{\color[rgb]{1,0,0}}%
      \expandafter\def\csname LT1\endcsname{\color[rgb]{0,1,0}}%
      \expandafter\def\csname LT2\endcsname{\color[rgb]{0,0,1}}%
      \expandafter\def\csname LT3\endcsname{\color[rgb]{1,0,1}}%
      \expandafter\def\csname LT4\endcsname{\color[rgb]{0,1,1}}%
      \expandafter\def\csname LT5\endcsname{\color[rgb]{1,1,0}}%
      \expandafter\def\csname LT6\endcsname{\color[rgb]{0,0,0}}%
      \expandafter\def\csname LT7\endcsname{\color[rgb]{1,0.3,0}}%
      \expandafter\def\csname LT8\endcsname{\color[rgb]{0.5,0.5,0.5}}%
    \else
      % gray
      \def\colorrgb#1{\color{black}}%
      \def\colorgray#1{\color[gray]{#1}}%
      \expandafter\def\csname LTw\endcsname{\color{white}}%
      \expandafter\def\csname LTb\endcsname{\color{black}}%
      \expandafter\def\csname LTa\endcsname{\color{black}}%
      \expandafter\def\csname LT0\endcsname{\color{black}}%
      \expandafter\def\csname LT1\endcsname{\color{black}}%
      \expandafter\def\csname LT2\endcsname{\color{black}}%
      \expandafter\def\csname LT3\endcsname{\color{black}}%
      \expandafter\def\csname LT4\endcsname{\color{black}}%
      \expandafter\def\csname LT5\endcsname{\color{black}}%
      \expandafter\def\csname LT6\endcsname{\color{black}}%
      \expandafter\def\csname LT7\endcsname{\color{black}}%
      \expandafter\def\csname LT8\endcsname{\color{black}}%
    \fi
  \fi
    \setlength{\unitlength}{0.0500bp}%
    \ifx\gptboxheight\undefined%
      \newlength{\gptboxheight}%
      \newlength{\gptboxwidth}%
      \newsavebox{\gptboxtext}%
    \fi%
    \setlength{\fboxrule}{0.5pt}%
    \setlength{\fboxsep}{1pt}%
\begin{picture}(4520.00,2540.00)%
    \gplgaddtomacro\gplbacktext{%
      \csname LTb\endcsname%%
      \put(202,329){\makebox(0,0)[r]{\strut{} }}%
      \csname LTb\endcsname%%
      \put(202,724){\makebox(0,0)[r]{\strut{} }}%
      \csname LTb\endcsname%%
      \put(202,1119){\makebox(0,0)[r]{\strut{} }}%
      \csname LTb\endcsname%%
      \put(202,1514){\makebox(0,0)[r]{\strut{} }}%
      \csname LTb\endcsname%%
      \put(202,1910){\makebox(0,0)[r]{\strut{} }}%
      \csname LTb\endcsname%%
      \put(202,2305){\makebox(0,0)[r]{\strut{} }}%
      \csname LTb\endcsname%%
      \put(283,68){\rotatebox{45}{\makebox(0,0){\strut{}1749}}}%
      \csname LTb\endcsname%%
      \put(717,68){\rotatebox{45}{\makebox(0,0){\strut{}1779}}}%
      \csname LTb\endcsname%%
      \put(1151,68){\rotatebox{45}{\makebox(0,0){\strut{}1809}}}%
      \csname LTb\endcsname%%
      \put(1585,68){\rotatebox{45}{\makebox(0,0){\strut{}1839}}}%
      \csname LTb\endcsname%%
      \put(2019,68){\rotatebox{45}{\makebox(0,0){\strut{}1869}}}%
      \csname LTb\endcsname%%
      \put(2453,68){\rotatebox{45}{\makebox(0,0){\strut{}1899}}}%
      \csname LTb\endcsname%%
      \put(2887,68){\rotatebox{45}{\makebox(0,0){\strut{}1930}}}%
      \csname LTb\endcsname%%
      \put(3321,68){\rotatebox{45}{\makebox(0,0){\strut{}1960}}}%
      \csname LTb\endcsname%%
      \put(3756,68){\rotatebox{45}{\makebox(0,0){\strut{}1990}}}%
      \csname LTb\endcsname%%
      \put(4190,68){\rotatebox{45}{\makebox(0,0){\strut{}2020}}}%
    }%
    \gplgaddtomacro\gplfronttext{%
    }%
    \gplbacktext
    \put(0,0){\includegraphics[width={226.00bp},height={127.00bp}]{../img/fourier/profilePlot}}%
    \gplfronttext
  \end{picture}%
\endgroup

%	\end{subfigure}
%	~
%	\begin{subfigure}{0.48\textwidth}
%		% GNUPLOT: LaTeX picture with Postscript
\begingroup
  \makeatletter
  \providecommand\color[2][]{%
    \GenericError{(gnuplot) \space\space\space\@spaces}{%
      Package color not loaded in conjunction with
      terminal option `colourtext'%
    }{See the gnuplot documentation for explanation.%
    }{Either use 'blacktext' in gnuplot or load the package
      color.sty in LaTeX.}%
    \renewcommand\color[2][]{}%
  }%
  \providecommand\includegraphics[2][]{%
    \GenericError{(gnuplot) \space\space\space\@spaces}{%
      Package graphicx or graphics not loaded%
    }{See the gnuplot documentation for explanation.%
    }{The gnuplot epslatex terminal needs graphicx.sty or graphics.sty.}%
    \renewcommand\includegraphics[2][]{}%
  }%
  \providecommand\rotatebox[2]{#2}%
  \@ifundefined{ifGPcolor}{%
    \newif\ifGPcolor
    \GPcolortrue
  }{}%
  \@ifundefined{ifGPblacktext}{%
    \newif\ifGPblacktext
    \GPblacktexttrue
  }{}%
  % define a \g@addto@macro without @ in the name:
  \let\gplgaddtomacro\g@addto@macro
  % define empty templates for all commands taking text:
  \gdef\gplbacktext{}%
  \gdef\gplfronttext{}%
  \makeatother
  \ifGPblacktext
    % no textcolor at all
    \def\colorrgb#1{}%
    \def\colorgray#1{}%
  \else
    % gray or color?
    \ifGPcolor
      \def\colorrgb#1{\color[rgb]{#1}}%
      \def\colorgray#1{\color[gray]{#1}}%
      \expandafter\def\csname LTw\endcsname{\color{white}}%
      \expandafter\def\csname LTb\endcsname{\color{black}}%
      \expandafter\def\csname LTa\endcsname{\color{black}}%
      \expandafter\def\csname LT0\endcsname{\color[rgb]{1,0,0}}%
      \expandafter\def\csname LT1\endcsname{\color[rgb]{0,1,0}}%
      \expandafter\def\csname LT2\endcsname{\color[rgb]{0,0,1}}%
      \expandafter\def\csname LT3\endcsname{\color[rgb]{1,0,1}}%
      \expandafter\def\csname LT4\endcsname{\color[rgb]{0,1,1}}%
      \expandafter\def\csname LT5\endcsname{\color[rgb]{1,1,0}}%
      \expandafter\def\csname LT6\endcsname{\color[rgb]{0,0,0}}%
      \expandafter\def\csname LT7\endcsname{\color[rgb]{1,0.3,0}}%
      \expandafter\def\csname LT8\endcsname{\color[rgb]{0.5,0.5,0.5}}%
    \else
      % gray
      \def\colorrgb#1{\color{black}}%
      \def\colorgray#1{\color[gray]{#1}}%
      \expandafter\def\csname LTw\endcsname{\color{white}}%
      \expandafter\def\csname LTb\endcsname{\color{black}}%
      \expandafter\def\csname LTa\endcsname{\color{black}}%
      \expandafter\def\csname LT0\endcsname{\color{black}}%
      \expandafter\def\csname LT1\endcsname{\color{black}}%
      \expandafter\def\csname LT2\endcsname{\color{black}}%
      \expandafter\def\csname LT3\endcsname{\color{black}}%
      \expandafter\def\csname LT4\endcsname{\color{black}}%
      \expandafter\def\csname LT5\endcsname{\color{black}}%
      \expandafter\def\csname LT6\endcsname{\color{black}}%
      \expandafter\def\csname LT7\endcsname{\color{black}}%
      \expandafter\def\csname LT8\endcsname{\color{black}}%
    \fi
  \fi
    \setlength{\unitlength}{0.0500bp}%
    \ifx\gptboxheight\undefined%
      \newlength{\gptboxheight}%
      \newlength{\gptboxwidth}%
      \newsavebox{\gptboxtext}%
    \fi%
    \setlength{\fboxrule}{0.5pt}%
    \setlength{\fboxsep}{1pt}%
\begin{picture}(4520.00,2540.00)%
    \gplgaddtomacro\gplbacktext{%
      \csname LTb\endcsname%%
      \put(445,1310){\makebox(0,0)[r]{\strut{}$ 10^{2}$}}%
      \csname LTb\endcsname%%
      \put(445,2376){\makebox(0,0)[r]{\strut{}$ 10^{3}$}}%
      \csname LTb\endcsname%%
      \put(1169,149){\makebox(0,0){\strut{}$ 10^{1}$}}%
      \csname LTb\endcsname%%
      \put(2784,149){\makebox(0,0){\strut{}$ 10^{2}$}}%
    }%
    \gplgaddtomacro\gplfronttext{%
      \csname LTb\endcsname%%
      \put(3630,691){\makebox(0,0)[r]{\strut{}1.31}}%
      \csname LTb\endcsname%%
      \put(3630,468){\makebox(0,0)[r]{\strut{}-0.02}}%
    }%
    \gplbacktext
    \put(0,0){\includegraphics[width={226.00bp},height={127.00bp}]{../img/fourier/DFAPlot}}%
    \gplfronttext
  \end{picture}%
\endgroup

%	\end{subfigure}
%	\caption{}
%\end{figure}

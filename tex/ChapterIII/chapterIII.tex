\section{Example: sun spots}
\graphicspath{{../sunSpots/img/}}
%copy paste wiki
Sunspots are temporary phenomena on the Sun's photosphere that appear as spots darker than the surrounding areas. They are regions of reduced surface temperature caused by concentrations of magnetic field flux that inhibit convection. Sunspots usually appear in pairs of opposite magnetic polarity.

%sunspot paper
The important feature of the sun’s outer regions is the existence of a reasonably strongmagnetic field. To the lowest order of approximation, the sun’s magnetic field is dipolar in character and is axisymmetric. The strength of the field at a typical point on the solar surface is approximately a few gauss. There is, however significant variation in this value and there are localized regions (called sunspots) in which the field can be much higher.Because of the symmetry of the twisted magnetic lines at the origin of sunspots, they are generally seen in pairs or in groups of pairs on both sides of the solar equator. 

\subsection{Raw data}
Database given by \cite{sidc}. 
%some information"

\begin{figure}[H]
	\centering
	\begin{subfigure}{\textwidth}
		% GNUPLOT: LaTeX picture with Postscript
\begingroup
  \makeatletter
  \providecommand\color[2][]{%
    \GenericError{(gnuplot) \space\space\space\@spaces}{%
      Package color not loaded in conjunction with
      terminal option `colourtext'%
    }{See the gnuplot documentation for explanation.%
    }{Either use 'blacktext' in gnuplot or load the package
      color.sty in LaTeX.}%
    \renewcommand\color[2][]{}%
  }%
  \providecommand\includegraphics[2][]{%
    \GenericError{(gnuplot) \space\space\space\@spaces}{%
      Package graphicx or graphics not loaded%
    }{See the gnuplot documentation for explanation.%
    }{The gnuplot epslatex terminal needs graphicx.sty or graphics.sty.}%
    \renewcommand\includegraphics[2][]{}%
  }%
  \providecommand\rotatebox[2]{#2}%
  \@ifundefined{ifGPcolor}{%
    \newif\ifGPcolor
    \GPcolortrue
  }{}%
  \@ifundefined{ifGPblacktext}{%
    \newif\ifGPblacktext
    \GPblacktexttrue
  }{}%
  % define a \g@addto@macro without @ in the name:
  \let\gplgaddtomacro\g@addto@macro
  % define empty templates for all commands taking text:
  \gdef\gplbacktext{}%
  \gdef\gplfronttext{}%
  \makeatother
  \ifGPblacktext
    % no textcolor at all
    \def\colorrgb#1{}%
    \def\colorgray#1{}%
  \else
    % gray or color?
    \ifGPcolor
      \def\colorrgb#1{\color[rgb]{#1}}%
      \def\colorgray#1{\color[gray]{#1}}%
      \expandafter\def\csname LTw\endcsname{\color{white}}%
      \expandafter\def\csname LTb\endcsname{\color{black}}%
      \expandafter\def\csname LTa\endcsname{\color{black}}%
      \expandafter\def\csname LT0\endcsname{\color[rgb]{1,0,0}}%
      \expandafter\def\csname LT1\endcsname{\color[rgb]{0,1,0}}%
      \expandafter\def\csname LT2\endcsname{\color[rgb]{0,0,1}}%
      \expandafter\def\csname LT3\endcsname{\color[rgb]{1,0,1}}%
      \expandafter\def\csname LT4\endcsname{\color[rgb]{0,1,1}}%
      \expandafter\def\csname LT5\endcsname{\color[rgb]{1,1,0}}%
      \expandafter\def\csname LT6\endcsname{\color[rgb]{0,0,0}}%
      \expandafter\def\csname LT7\endcsname{\color[rgb]{1,0.3,0}}%
      \expandafter\def\csname LT8\endcsname{\color[rgb]{0.5,0.5,0.5}}%
    \else
      % gray
      \def\colorrgb#1{\color{black}}%
      \def\colorgray#1{\color[gray]{#1}}%
      \expandafter\def\csname LTw\endcsname{\color{white}}%
      \expandafter\def\csname LTb\endcsname{\color{black}}%
      \expandafter\def\csname LTa\endcsname{\color{black}}%
      \expandafter\def\csname LT0\endcsname{\color{black}}%
      \expandafter\def\csname LT1\endcsname{\color{black}}%
      \expandafter\def\csname LT2\endcsname{\color{black}}%
      \expandafter\def\csname LT3\endcsname{\color{black}}%
      \expandafter\def\csname LT4\endcsname{\color{black}}%
      \expandafter\def\csname LT5\endcsname{\color{black}}%
      \expandafter\def\csname LT6\endcsname{\color{black}}%
      \expandafter\def\csname LT7\endcsname{\color{black}}%
      \expandafter\def\csname LT8\endcsname{\color{black}}%
    \fi
  \fi
    \setlength{\unitlength}{0.0500bp}%
    \ifx\gptboxheight\undefined%
      \newlength{\gptboxheight}%
      \newlength{\gptboxwidth}%
      \newsavebox{\gptboxtext}%
    \fi%
    \setlength{\fboxrule}{0.5pt}%
    \setlength{\fboxsep}{1pt}%
\begin{picture}(4520.00,2540.00)%
    \gplgaddtomacro\gplbacktext{%
      \csname LTb\endcsname%%
      \put(364,298){\makebox(0,0)[r]{\strut{}$0$}}%
      \csname LTb\endcsname%%
      \put(364,655){\makebox(0,0)[r]{\strut{}$100$}}%
      \csname LTb\endcsname%%
      \put(364,1011){\makebox(0,0)[r]{\strut{}$200$}}%
      \csname LTb\endcsname%%
      \put(364,1368){\makebox(0,0)[r]{\strut{}$300$}}%
      \csname LTb\endcsname%%
      \put(364,1725){\makebox(0,0)[r]{\strut{}$400$}}%
      \csname LTb\endcsname%%
      \put(364,2082){\makebox(0,0)[r]{\strut{}$500$}}%
      \csname LTb\endcsname%%
      \put(445,68){\rotatebox{45}{\makebox(0,0){\strut{}1818}}}%
      \csname LTb\endcsname%%
      \put(1001,68){\rotatebox{45}{\makebox(0,0){\strut{}1848}}}%
      \csname LTb\endcsname%%
      \put(1405,68){\rotatebox{45}{\makebox(0,0){\strut{}1869}}}%
      \csname LTb\endcsname%%
      \put(1803,68){\rotatebox{45}{\makebox(0,0){\strut{}1891}}}%
      \csname LTb\endcsname%%
      \put(2201,68){\rotatebox{45}{\makebox(0,0){\strut{}1912}}}%
      \csname LTb\endcsname%%
      \put(2598,68){\rotatebox{45}{\makebox(0,0){\strut{}1934}}}%
      \csname LTb\endcsname%%
      \put(2996,68){\rotatebox{45}{\makebox(0,0){\strut{}1955}}}%
      \csname LTb\endcsname%%
      \put(3394,68){\rotatebox{45}{\makebox(0,0){\strut{}1977}}}%
      \csname LTb\endcsname%%
      \put(3792,68){\rotatebox{45}{\makebox(0,0){\strut{}1998}}}%
      \csname LTb\endcsname%%
      \put(4190,68){\rotatebox{45}{\makebox(0,0){\strut{}2020}}}%
    }%
    \gplgaddtomacro\gplfronttext{%
    }%
    \gplbacktext
    \put(0,0){\includegraphics[width={226.00bp},height={127.00bp}]{../img/daily/dataPlot}}%
    \gplfronttext
  \end{picture}%
\endgroup

	\end{subfigure}

	\begin{subfigure}{\textwidth}
		% GNUPLOT: LaTeX picture with Postscript
\begingroup
  \makeatletter
  \providecommand\color[2][]{%
    \GenericError{(gnuplot) \space\space\space\@spaces}{%
      Package color not loaded in conjunction with
      terminal option `colourtext'%
    }{See the gnuplot documentation for explanation.%
    }{Either use 'blacktext' in gnuplot or load the package
      color.sty in LaTeX.}%
    \renewcommand\color[2][]{}%
  }%
  \providecommand\includegraphics[2][]{%
    \GenericError{(gnuplot) \space\space\space\@spaces}{%
      Package graphicx or graphics not loaded%
    }{See the gnuplot documentation for explanation.%
    }{The gnuplot epslatex terminal needs graphicx.sty or graphics.sty.}%
    \renewcommand\includegraphics[2][]{}%
  }%
  \providecommand\rotatebox[2]{#2}%
  \@ifundefined{ifGPcolor}{%
    \newif\ifGPcolor
    \GPcolortrue
  }{}%
  \@ifundefined{ifGPblacktext}{%
    \newif\ifGPblacktext
    \GPblacktexttrue
  }{}%
  % define a \g@addto@macro without @ in the name:
  \let\gplgaddtomacro\g@addto@macro
  % define empty templates for all commands taking text:
  \gdef\gplbacktext{}%
  \gdef\gplfronttext{}%
  \makeatother
  \ifGPblacktext
    % no textcolor at all
    \def\colorrgb#1{}%
    \def\colorgray#1{}%
  \else
    % gray or color?
    \ifGPcolor
      \def\colorrgb#1{\color[rgb]{#1}}%
      \def\colorgray#1{\color[gray]{#1}}%
      \expandafter\def\csname LTw\endcsname{\color{white}}%
      \expandafter\def\csname LTb\endcsname{\color{black}}%
      \expandafter\def\csname LTa\endcsname{\color{black}}%
      \expandafter\def\csname LT0\endcsname{\color[rgb]{1,0,0}}%
      \expandafter\def\csname LT1\endcsname{\color[rgb]{0,1,0}}%
      \expandafter\def\csname LT2\endcsname{\color[rgb]{0,0,1}}%
      \expandafter\def\csname LT3\endcsname{\color[rgb]{1,0,1}}%
      \expandafter\def\csname LT4\endcsname{\color[rgb]{0,1,1}}%
      \expandafter\def\csname LT5\endcsname{\color[rgb]{1,1,0}}%
      \expandafter\def\csname LT6\endcsname{\color[rgb]{0,0,0}}%
      \expandafter\def\csname LT7\endcsname{\color[rgb]{1,0.3,0}}%
      \expandafter\def\csname LT8\endcsname{\color[rgb]{0.5,0.5,0.5}}%
    \else
      % gray
      \def\colorrgb#1{\color{black}}%
      \def\colorgray#1{\color[gray]{#1}}%
      \expandafter\def\csname LTw\endcsname{\color{white}}%
      \expandafter\def\csname LTb\endcsname{\color{black}}%
      \expandafter\def\csname LTa\endcsname{\color{black}}%
      \expandafter\def\csname LT0\endcsname{\color{black}}%
      \expandafter\def\csname LT1\endcsname{\color{black}}%
      \expandafter\def\csname LT2\endcsname{\color{black}}%
      \expandafter\def\csname LT3\endcsname{\color{black}}%
      \expandafter\def\csname LT4\endcsname{\color{black}}%
      \expandafter\def\csname LT5\endcsname{\color{black}}%
      \expandafter\def\csname LT6\endcsname{\color{black}}%
      \expandafter\def\csname LT7\endcsname{\color{black}}%
      \expandafter\def\csname LT8\endcsname{\color{black}}%
    \fi
  \fi
    \setlength{\unitlength}{0.0500bp}%
    \ifx\gptboxheight\undefined%
      \newlength{\gptboxheight}%
      \newlength{\gptboxwidth}%
      \newsavebox{\gptboxtext}%
    \fi%
    \setlength{\fboxrule}{0.5pt}%
    \setlength{\fboxsep}{1pt}%
\begin{picture}(9060.00,5100.00)%
    \gplgaddtomacro\gplbacktext{%
      \csname LTb\endcsname%%
      \put(708,652){\makebox(0,0)[r]{\strut{}$0$}}%
      \csname LTb\endcsname%%
      \put(708,1087){\makebox(0,0)[r]{\strut{}$50$}}%
      \csname LTb\endcsname%%
      \put(708,1523){\makebox(0,0)[r]{\strut{}$100$}}%
      \csname LTb\endcsname%%
      \put(708,1958){\makebox(0,0)[r]{\strut{}$150$}}%
      \csname LTb\endcsname%%
      \put(708,2394){\makebox(0,0)[r]{\strut{}$200$}}%
      \csname LTb\endcsname%%
      \put(708,2829){\makebox(0,0)[r]{\strut{}$250$}}%
      \csname LTb\endcsname%%
      \put(708,3265){\makebox(0,0)[r]{\strut{}$300$}}%
      \csname LTb\endcsname%%
      \put(708,3700){\makebox(0,0)[r]{\strut{}$350$}}%
      \csname LTb\endcsname%%
      \put(708,4136){\makebox(0,0)[r]{\strut{}$400$}}%
      \csname LTb\endcsname%%
      \put(820,448){\makebox(0,0){\strut{}1749}}%
      \csname LTb\endcsname%%
      \put(1695,448){\makebox(0,0){\strut{}1779}}%
      \csname LTb\endcsname%%
      \put(2571,448){\makebox(0,0){\strut{}1809}}%
      \csname LTb\endcsname%%
      \put(3446,448){\makebox(0,0){\strut{}1839}}%
      \csname LTb\endcsname%%
      \put(4321,448){\makebox(0,0){\strut{}1869}}%
      \csname LTb\endcsname%%
      \put(5197,448){\makebox(0,0){\strut{}1899}}%
      \csname LTb\endcsname%%
      \put(6072,448){\makebox(0,0){\strut{}1930}}%
      \csname LTb\endcsname%%
      \put(6947,448){\makebox(0,0){\strut{}1960}}%
      \csname LTb\endcsname%%
      \put(7823,448){\makebox(0,0){\strut{}1990}}%
      \csname LTb\endcsname%%
      \put(8698,448){\makebox(0,0){\strut{}2020}}%
    }%
    \gplgaddtomacro\gplfronttext{%
      \csname LTb\endcsname%%
      \put(186,2559){\rotatebox{-270}{\makebox(0,0){\strut{}Count}}}%
      \csname LTb\endcsname%%
      \put(4761,142){\makebox(0,0){\strut{}Time [year]}}%
      \csname LTb\endcsname%%
      \put(4761,4773){\makebox(0,0){\strut{}Monthly observed sun spots number}}%
    }%
    \gplbacktext
    \put(0,0){\includegraphics[width={453.00bp},height={255.00bp}]{../img/monthly/dataPlot}}%
    \gplfronttext
  \end{picture}%
\endgroup

	\end{subfigure}
	\caption{Raw data $x_n$}
\end{figure}

\subsection{Data Analysis}

\subsubsection{Profile}


\begin{figure}[H]
	\centering
	\begin{subfigure}{\textwidth}
		% GNUPLOT: LaTeX picture with Postscript
\begingroup
  \makeatletter
  \providecommand\color[2][]{%
    \GenericError{(gnuplot) \space\space\space\@spaces}{%
      Package color not loaded in conjunction with
      terminal option `colourtext'%
    }{See the gnuplot documentation for explanation.%
    }{Either use 'blacktext' in gnuplot or load the package
      color.sty in LaTeX.}%
    \renewcommand\color[2][]{}%
  }%
  \providecommand\includegraphics[2][]{%
    \GenericError{(gnuplot) \space\space\space\@spaces}{%
      Package graphicx or graphics not loaded%
    }{See the gnuplot documentation for explanation.%
    }{The gnuplot epslatex terminal needs graphicx.sty or graphics.sty.}%
    \renewcommand\includegraphics[2][]{}%
  }%
  \providecommand\rotatebox[2]{#2}%
  \@ifundefined{ifGPcolor}{%
    \newif\ifGPcolor
    \GPcolortrue
  }{}%
  \@ifundefined{ifGPblacktext}{%
    \newif\ifGPblacktext
    \GPblacktexttrue
  }{}%
  % define a \g@addto@macro without @ in the name:
  \let\gplgaddtomacro\g@addto@macro
  % define empty templates for all commands taking text:
  \gdef\gplbacktext{}%
  \gdef\gplfronttext{}%
  \makeatother
  \ifGPblacktext
    % no textcolor at all
    \def\colorrgb#1{}%
    \def\colorgray#1{}%
  \else
    % gray or color?
    \ifGPcolor
      \def\colorrgb#1{\color[rgb]{#1}}%
      \def\colorgray#1{\color[gray]{#1}}%
      \expandafter\def\csname LTw\endcsname{\color{white}}%
      \expandafter\def\csname LTb\endcsname{\color{black}}%
      \expandafter\def\csname LTa\endcsname{\color{black}}%
      \expandafter\def\csname LT0\endcsname{\color[rgb]{1,0,0}}%
      \expandafter\def\csname LT1\endcsname{\color[rgb]{0,1,0}}%
      \expandafter\def\csname LT2\endcsname{\color[rgb]{0,0,1}}%
      \expandafter\def\csname LT3\endcsname{\color[rgb]{1,0,1}}%
      \expandafter\def\csname LT4\endcsname{\color[rgb]{0,1,1}}%
      \expandafter\def\csname LT5\endcsname{\color[rgb]{1,1,0}}%
      \expandafter\def\csname LT6\endcsname{\color[rgb]{0,0,0}}%
      \expandafter\def\csname LT7\endcsname{\color[rgb]{1,0.3,0}}%
      \expandafter\def\csname LT8\endcsname{\color[rgb]{0.5,0.5,0.5}}%
    \else
      % gray
      \def\colorrgb#1{\color{black}}%
      \def\colorgray#1{\color[gray]{#1}}%
      \expandafter\def\csname LTw\endcsname{\color{white}}%
      \expandafter\def\csname LTb\endcsname{\color{black}}%
      \expandafter\def\csname LTa\endcsname{\color{black}}%
      \expandafter\def\csname LT0\endcsname{\color{black}}%
      \expandafter\def\csname LT1\endcsname{\color{black}}%
      \expandafter\def\csname LT2\endcsname{\color{black}}%
      \expandafter\def\csname LT3\endcsname{\color{black}}%
      \expandafter\def\csname LT4\endcsname{\color{black}}%
      \expandafter\def\csname LT5\endcsname{\color{black}}%
      \expandafter\def\csname LT6\endcsname{\color{black}}%
      \expandafter\def\csname LT7\endcsname{\color{black}}%
      \expandafter\def\csname LT8\endcsname{\color{black}}%
    \fi
  \fi
    \setlength{\unitlength}{0.0500bp}%
    \ifx\gptboxheight\undefined%
      \newlength{\gptboxheight}%
      \newlength{\gptboxwidth}%
      \newsavebox{\gptboxtext}%
    \fi%
    \setlength{\fboxrule}{0.5pt}%
    \setlength{\fboxsep}{1pt}%
\begin{picture}(9060.00,5100.00)%
    \gplgaddtomacro\gplbacktext{%
      \csname LTb\endcsname%%
      \put(1064,849){\makebox(0,0)[r]{\strut{}$ -4 \times10^{5}$}}%
      \csname LTb\endcsname%%
      \put(1064,1395){\makebox(0,0)[r]{\strut{}$ -3 \times10^{5}$}}%
      \csname LTb\endcsname%%
      \put(1064,1941){\makebox(0,0)[r]{\strut{}$ -2 \times10^{5}$}}%
      \csname LTb\endcsname%%
      \put(1064,2486){\makebox(0,0)[r]{\strut{}$ -1 \times10^{5}$}}%
      \csname LTb\endcsname%%
      \put(1064,3032){\makebox(0,0)[r]{\strut{}$  0 \times10^{0}$}}%
      \csname LTb\endcsname%%
      \put(1064,3578){\makebox(0,0)[r]{\strut{}$  1 \times10^{5}$}}%
      \csname LTb\endcsname%%
      \put(1064,4124){\makebox(0,0)[r]{\strut{}$  2 \times10^{5}$}}%
      \csname LTb\endcsname%%
      \put(1176,448){\makebox(0,0){\strut{}1818}}%
      \csname LTb\endcsname%%
      \put(2294,448){\makebox(0,0){\strut{}1848}}%
      \csname LTb\endcsname%%
      \put(3105,448){\makebox(0,0){\strut{}1869}}%
      \csname LTb\endcsname%%
      \put(3905,448){\makebox(0,0){\strut{}1891}}%
      \csname LTb\endcsname%%
      \put(4705,448){\makebox(0,0){\strut{}1912}}%
      \csname LTb\endcsname%%
      \put(5504,448){\makebox(0,0){\strut{}1934}}%
      \csname LTb\endcsname%%
      \put(6304,448){\makebox(0,0){\strut{}1955}}%
      \csname LTb\endcsname%%
      \put(7104,448){\makebox(0,0){\strut{}1977}}%
      \csname LTb\endcsname%%
      \put(7903,448){\makebox(0,0){\strut{}1998}}%
      \csname LTb\endcsname%%
      \put(8703,448){\makebox(0,0){\strut{}2020}}%
    }%
    \gplgaddtomacro\gplfronttext{%
      \csname LTb\endcsname%%
      \put(4939,142){\makebox(0,0){\strut{}Time [year]}}%
      \csname LTb\endcsname%%
      \put(4939,4773){\makebox(0,0){\strut{}profile function for sun spots number}}%
    }%
    \gplbacktext
    \put(0,0){\includegraphics[width={453.00bp},height={255.00bp}]{../img/daily/profilePlot}}%
    \gplfronttext
  \end{picture}%
\endgroup

	\end{subfigure}

	\begin{subfigure}{\textwidth}
		% GNUPLOT: LaTeX picture with Postscript
\begingroup
  \makeatletter
  \providecommand\color[2][]{%
    \GenericError{(gnuplot) \space\space\space\@spaces}{%
      Package color not loaded in conjunction with
      terminal option `colourtext'%
    }{See the gnuplot documentation for explanation.%
    }{Either use 'blacktext' in gnuplot or load the package
      color.sty in LaTeX.}%
    \renewcommand\color[2][]{}%
  }%
  \providecommand\includegraphics[2][]{%
    \GenericError{(gnuplot) \space\space\space\@spaces}{%
      Package graphicx or graphics not loaded%
    }{See the gnuplot documentation for explanation.%
    }{The gnuplot epslatex terminal needs graphicx.sty or graphics.sty.}%
    \renewcommand\includegraphics[2][]{}%
  }%
  \providecommand\rotatebox[2]{#2}%
  \@ifundefined{ifGPcolor}{%
    \newif\ifGPcolor
    \GPcolortrue
  }{}%
  \@ifundefined{ifGPblacktext}{%
    \newif\ifGPblacktext
    \GPblacktexttrue
  }{}%
  % define a \g@addto@macro without @ in the name:
  \let\gplgaddtomacro\g@addto@macro
  % define empty templates for all commands taking text:
  \gdef\gplbacktext{}%
  \gdef\gplfronttext{}%
  \makeatother
  \ifGPblacktext
    % no textcolor at all
    \def\colorrgb#1{}%
    \def\colorgray#1{}%
  \else
    % gray or color?
    \ifGPcolor
      \def\colorrgb#1{\color[rgb]{#1}}%
      \def\colorgray#1{\color[gray]{#1}}%
      \expandafter\def\csname LTw\endcsname{\color{white}}%
      \expandafter\def\csname LTb\endcsname{\color{black}}%
      \expandafter\def\csname LTa\endcsname{\color{black}}%
      \expandafter\def\csname LT0\endcsname{\color[rgb]{1,0,0}}%
      \expandafter\def\csname LT1\endcsname{\color[rgb]{0,1,0}}%
      \expandafter\def\csname LT2\endcsname{\color[rgb]{0,0,1}}%
      \expandafter\def\csname LT3\endcsname{\color[rgb]{1,0,1}}%
      \expandafter\def\csname LT4\endcsname{\color[rgb]{0,1,1}}%
      \expandafter\def\csname LT5\endcsname{\color[rgb]{1,1,0}}%
      \expandafter\def\csname LT6\endcsname{\color[rgb]{0,0,0}}%
      \expandafter\def\csname LT7\endcsname{\color[rgb]{1,0.3,0}}%
      \expandafter\def\csname LT8\endcsname{\color[rgb]{0.5,0.5,0.5}}%
    \else
      % gray
      \def\colorrgb#1{\color{black}}%
      \def\colorgray#1{\color[gray]{#1}}%
      \expandafter\def\csname LTw\endcsname{\color{white}}%
      \expandafter\def\csname LTb\endcsname{\color{black}}%
      \expandafter\def\csname LTa\endcsname{\color{black}}%
      \expandafter\def\csname LT0\endcsname{\color{black}}%
      \expandafter\def\csname LT1\endcsname{\color{black}}%
      \expandafter\def\csname LT2\endcsname{\color{black}}%
      \expandafter\def\csname LT3\endcsname{\color{black}}%
      \expandafter\def\csname LT4\endcsname{\color{black}}%
      \expandafter\def\csname LT5\endcsname{\color{black}}%
      \expandafter\def\csname LT6\endcsname{\color{black}}%
      \expandafter\def\csname LT7\endcsname{\color{black}}%
      \expandafter\def\csname LT8\endcsname{\color{black}}%
    \fi
  \fi
    \setlength{\unitlength}{0.0500bp}%
    \ifx\gptboxheight\undefined%
      \newlength{\gptboxheight}%
      \newlength{\gptboxwidth}%
      \newsavebox{\gptboxtext}%
    \fi%
    \setlength{\fboxrule}{0.5pt}%
    \setlength{\fboxsep}{1pt}%
\begin{picture}(9060.00,5100.00)%
    \gplgaddtomacro\gplbacktext{%
      \csname LTb\endcsname%%
      \put(1064,1376){\makebox(0,0)[r]{\strut{}$ -1 \times10^{4}$}}%
      \csname LTb\endcsname%%
      \put(1064,2130){\makebox(0,0)[r]{\strut{}$ -5 \times10^{3}$}}%
      \csname LTb\endcsname%%
      \put(1064,2883){\makebox(0,0)[r]{\strut{}$  0 \times10^{0}$}}%
      \csname LTb\endcsname%%
      \put(1064,3636){\makebox(0,0)[r]{\strut{}$  5 \times10^{3}$}}%
      \csname LTb\endcsname%%
      \put(1064,4389){\makebox(0,0)[r]{\strut{}$  1 \times10^{4}$}}%
      \csname LTb\endcsname%%
      \put(1176,448){\makebox(0,0){\strut{}1749}}%
      \csname LTb\endcsname%%
      \put(2012,448){\makebox(0,0){\strut{}1779}}%
      \csname LTb\endcsname%%
      \put(2848,448){\makebox(0,0){\strut{}1809}}%
      \csname LTb\endcsname%%
      \put(3683,448){\makebox(0,0){\strut{}1839}}%
      \csname LTb\endcsname%%
      \put(4519,448){\makebox(0,0){\strut{}1869}}%
      \csname LTb\endcsname%%
      \put(5355,448){\makebox(0,0){\strut{}1899}}%
      \csname LTb\endcsname%%
      \put(6191,448){\makebox(0,0){\strut{}1930}}%
      \csname LTb\endcsname%%
      \put(7027,448){\makebox(0,0){\strut{}1960}}%
      \csname LTb\endcsname%%
      \put(7863,448){\makebox(0,0){\strut{}1990}}%
      \csname LTb\endcsname%%
      \put(8698,448){\makebox(0,0){\strut{}2020}}%
    }%
    \gplgaddtomacro\gplfronttext{%
      \csname LTb\endcsname%%
      \put(4939,142){\makebox(0,0){\strut{}Time [year]}}%
      \csname LTb\endcsname%%
      \put(4939,4773){\makebox(0,0){\strut{}profile function for sun spots number}}%
    }%
    \gplbacktext
    \put(0,0){\includegraphics[width={453.00bp},height={255.00bp}]{../img/monthly/profilePlot}}%
    \gplfronttext
  \end{picture}%
\endgroup

	\end{subfigure}
	\caption{Profile $y_n$}
\end{figure}

\subsubsection{DFA}
Calculate the DFA for specific $\tau$

\subsubsection{Hurst exponent}
insert table here


\subsection{Conclusion}